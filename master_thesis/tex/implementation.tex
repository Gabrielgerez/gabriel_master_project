\makeatletter
\def\input@path{{../}}
\makeatother
\documentclass[../master_thesis.tex]{subfiles}
\begin{document}
\chapter{Implementation of The Generalized Poisson Equation}\label{chap:implementation}
\section{Generalized Poisson equation on Multiwavelet basis}
In the following chapter we will outline our implementation of the \ac{GPE}
for solvent systems. We took Fosso--Tande's and Harrison's
paper in \cite{FossoTande:2013ka} as a starting point. There they defined
how to solve a \ac{GPE} that included a dielectric function which was analytical
through the cavity boundary. This model was explained with more detail in chapter \ref{chap:Solvent_effect} and
it will not be explained again.

\subsection{Cavity Function}
The first step was to create a cavity function as in \cite{FossoTande:2013ka}.
This was done by creating a cavity object that stored the coordinates of nuclei
$\rvec_I$ and their characteristic radii $R_I$. When a point $\rvec$ is evaluated
in the Cavity function it will return $0$ if it is outside and $1$ if
it is inside  and will have a sigmoidal shape at the boundary  of the
Interlocking spheres cavity defined by the nuclei coordinates and their radii.
Additionally we wanted to be able to change the width of the
boundary with a parameter $\sigma$. The structure of the cavity object is as
in Algorithm \ref{alg:Cavity}

\begin{algorithm}
  \caption{Cavity object}\label{alg:Cavity}
  \begin{algorithmic}
    \STATE \underline{\textbf{Initialize $C(\rvec)$}}
    \STATE \textbf{Input :} Molecular Coordinates, Radii, Width
    \STATE $\sigma \leftarrow $ Width
    \STATE Set $C_{tot}(\rvec) = 1$
    \FOR{ All Nuclei $I$}
     \STATE $\rvec_I \leftarrow $ Molecular Coordinate$_I$
     \STATE $R_I \leftarrow \text{Radius}_I$
    \ENDFOR
    \STATE
    \STATE \underline{\textbf{When Evaluating $C(\rvec)$}}
    \STATE \textbf{Input: } $\rvec$
    \FOR{All Nuclei I}
      \STATE $s_I(\rvec) = \abs{\rvec - \rvec_I} - R_I$
      \STATE $\Theta_I(\rvec) = \frac{1}{2}\left(1 + \erf\left(\frac{s_I(\rvec)}{\sigma}\right)\right)$
      \STATE $ C_I(\rvec) = 1 -\Theta_I(\rvec) $
      \STATE $ C_{tot}(\rvec) \leftarrow C_{tot}(\rvec) \cdot (1 - C_{I}(\rvec))$
    \ENDFOR
    \RETURN $ 1 - C_{tot}(\rvec)$
  \end{algorithmic}
\end{algorithm}
This Object was implemented as a derived class of \verb!RepresentableFunction!,
a C++ class in \mrchem.
\subsection{Dielectric Function}
In Equations \ref{eq:dielfunclin} and \ref{eq:dielfuncexp} we defined both a
linear $\epslin$ and exponential $\epsexp$ dielectric function. These were implemented as presented
in those two equations, with no changes to them, except for minor syntax related
adjustments. Their $\log$ derivatives had to be implemented separately,
this was done in order to have better convergence \cite{FossoTande:2013ka}. The
$\log$ derivative of the exponential dielectric function is represented as the
derivative of the Cavity multiplied by a constant. The cavity of water as implemented
above can be seen in Figure \ref{fig:watcav} as a slice through the $xz$ plane.

\begin{figure}[ht]
  \includegraphics[width=\linewidth]{img/Figure_1-1.png}
  \caption[Interlocking spheres cavity for water]{Interlocking spheres cavity function slice for water as implemented in algorithm \ref{alg:Cavity}}
  \label{fig:watcav}
\end{figure}
\clearpage

\subsection{Effective volume charge distribution}
When Running a \ac{SCF} calculation one calculates with electron densities in order
to solve an electron \SE as is consistent with the \ac{BO} approximation
\cite{Cramer:2004, Konishi:2009}, whereas the \ac{GPE} needs the total molecular
density $\rho_{tot}$. This is computed as a sum of the electron
density $\rho_{el}$ and the nuclear density $\rho_{nuc}$ based on the
geometry and charge of the nuclei
\begin{equation}
  \rho_{tot} = \rho_{el} + \rho_{nuc}
\end{equation}.
Algorithm \ref{alg:rhonuc} shows how we calculated the nuclear
density $\rho_{nuc}$ for the total density by use of point charges represented as
Gaussian functions centered at each nucleus.
\begin{algorithm}
  \caption{Nuclear charge density}\label{alg:rhonuc}
  \begin{algorithmic}
    \STATE \textbf{Input :} Nuclear coordinates, $Z_I$
    \STATE $\alpha = 1000$
    \STATE $\beta = (\frac{\alpha}{\pi})^{\frac{3}{2}} \cdot Z_I$
    \STATE $\rvec_I(\rvec) =$ Nuclear coordinate$_I$
    \STATE Set $\rho_{nuc} = 0$
    \FOR{All nuclei $I$}
    \STATE $\rho_{I}(\rvec) = \beta e^{-\alpha\cdot\lVert\rvec -\rvec_I\rVert^2}$
    \STATE $\rho_{nuc(\rvec)} = \rho_{nuc(\rvec)} + \rho_{nuc}^{(I)}(\rvec)$
    \ENDFOR
    \RETURN $\rho_{nuc}(\rvec)$
  \end{algorithmic}
\end{algorithm}

Where $\alpha$ represents the width of the Gaussian function, $\beta$ is the normalization
constant multiplied with the charge, $\rvec_I$ is the position of the nucleus and
$\rho_{nuc}^{(I)}$ is the Gaussian representing the a point charge at the nucleus.

We then go on to create the effective volume charge distribution by the following equation.
\begin{align}
    \begin{split}
      \rho_{eff} = \frac{1 - \epsilon}{\epsilon}\rho_{tot}\\
    \end{split}
\end{align}
This differs from Fosso-Tande's effective charge distribution in that we incorporate
the subtraction of $U_{vac}$ to it from the start \cite{FossoTande:2013ka}. This
way we will be solving directly for the reaction field potential $U_r$.
The alternative is to instead compute the total potential $U$ and then subtract
the gas phase potential to get the Reaction potential as in \cite{FossoTande:2013ka}.
Calculating the Reaction potential with the latter method brings loss of accuracy
by adding numerical noise.

\subsection{Surface Charge distribution}
Given the total interaction potential of $U$ of a solvation system with
dielectric function $\epsilon$ defined with cavity $C$, we compute the surface charge distribution
$\gamma_s$ as shown in algorithm \ref{alg:gamma}.
\begin{algorithm}
  \caption{Surface charge distribution}\label{alg:gamma}
  \begin{algorithmic}
    \STATE $\gamma_s[U, \epsilon[C]]$:
    \STATE \textbf{Input :} $U$ potential, $\epsilon[C]$ dielectric function
    \IF{$\epsilon$ is exponential}
      \STATE $k = \frac{1}{4\pi}\log\frac{\epso}{\epsinf}$
      \RETURN $k \nabla C \cdot \nabla U$
    \ELSIF{$\epsilon$ is linear}
      \STATE $k = \frac{1}{4\pi}(\epso-\epsinf)$
      \RETURN $\frac{k}{\epsilon} \nabla C \cdot \nabla U$
    \ENDIF
  \end{algorithmic}
\end{algorithm}

In algorithm \ref{alg:gamma} we use the definition of the log derivative for both $\epslin$ and $\epsexp$
from equations \ref{eq:logderepsexp} and \ref{eq:logderepslin}. This is the same
as just writing $$\gamma_s = \frac{1}{4\pi}\frac{\nabla\epsilon\nabla U}{\epsilon},$$
which is the same equation as in \ref{eq:effrhogamma}.

\subsection{The iterative \ac{SCRF} method}
In this method we Compute the Reaction field of the solvation system by iteration.
Here we follow the equation for the Reaction field potential in section \ref{solmw}.
In that section we see that, in order to compute the reaction potential, we need
to compute the surface charge distribution of the reaction potential. This
paradox is resolved by an iterative process. In this section I will explain how
this process works.

The first potential that is calculated will be a gas phase potential $U_{vac}$ this
potential is found by applying the Poisson operator $\hat{\mathscr{P}}$ defined
in Equation \ref{eq:Poissonopmw} on only the total charge density $\rho_{tot}$.
This is then used to make the zeroth Surface charge distribution $\gamma_s^{(0)}$.
This zeroth gamma is then added to the effective volume charge distribution $\rho_{eff}$
and the Poisson operator is applied to the sum in order to get the reaction potential.
This reaction potential is then used next iteration if the \ac{SCRF} to compute
the first surface charge distribution, which is then used to compute the next
and so on. This is done until the norm of the difference between the previous
reaction potential and the new one is less than a user defined precision. This is shown in
Algorithm \ref{alg:Reactionfield}.
\begin{algorithm}
  \caption{\ac{SCRF} iterative method}\label{alg:Reactionfield}
  \begin{algorithmic}
    \STATE \underline{\textbf{Zeroth iteration:}}
    \STATE $U_{vac} \leftarrow \hat{\mathscr{P}}(\rho_{tot})$
    \STATE $\gamma_s^{(0)} \leftarrow \gamma_s\big[U_{vac}, \epsilon\big]$
    \STATE $U_r^{(1)} \leftarrow \hat{\mathscr{P}}(\rho_{eff} + \gamma_s^{(0)})$
    \STATE \underline{\textbf{n-th iteration:}}
    \STATE \textbf{Input :} precision
    \STATE error = 10
    \WHILE{error $\geqslant$ precision}
      \STATE $\gamma_s^{(n)} \leftarrow \gamma_s\big[(U_{vac}+ U_r^{(n)}), \epsilon\big]$
      \STATE $U_r^{(n+1)} \leftarrow \hat{\mathscr{P}} (\rho_{eff} + \gamma_s^{(n)})$
      \STATE error = $\lVert U_r^{(n+1)} - U_r^{(n)} \rVert$
      \STATE $U_r^{(n)} \leftarrow U_r^{(n+1)}$
    \ENDWHILE
    \RETURN $U_r^{(n)}$
  \end{algorithmic}
\end{algorithm}
The \ac{SCRF} can then be implemented by itself inside a \ac{SCF} cycle.
The converged Reaction potential for water in a water solvent can be seen in
Figure \ref{fig:watpots}.

\begin{figure}[h!]
  \centering
  \begin{subfigure}[b]{0.49\linewidth}
    \includegraphics[width=\linewidth]{img/Urpot5.png}
  \end{subfigure}
  \begin{subfigure}[b]{0.49\linewidth}
    \includegraphics[width=\linewidth]{img/Urpot2.png}
  \end{subfigure}
  \begin{subfigure}[b]{0.49\linewidth}
    \includegraphics[width=\linewidth]{img/Urpot3.png}
  \end{subfigure}
  \begin{subfigure}[b]{0.49\linewidth}
    \includegraphics[width=\linewidth]{img/Urpot4.png}
  \end{subfigure}
  \begin{subfigure}[b]{\linewidth}
    \includegraphics[width=\linewidth]{img/Urpot1.png}
  \end{subfigure}
  \caption{Reaction potential of water across the XZ plane}
  \label{fig:watpots}
\end{figure}

\section{Variational implementation}
The variational implementation is very similar to the iterational one described.
All the steps outlined above are the same during the first \ac{SCF} cycle. Except
that in the end, we compute the next $\gamma_s^{(n)}$ with the converged $U_r$ before
moving on to the next \ac{SCF} cycle. This $\gamma_s^{(n)}$ is then used in the next cycle
to compute only one iteration of the \ac{SCRF} without checking the error against the
precision. After the one iteration we use the new $U_R$ to compute the $\gamma_s^{(n+1)}$
for the next iteration. If one is using a convergence accelerator one can make
use of the old surface charge distribution $\gamma_s^{(n)}$ and the new one just computed
$\gamma_s^{(n+1)}$ to accelerate the convergence of the Reaction field potential.
This is outlined in algorithm \ref{alg:Reactionfieldvar}.
\begin{algorithm}
  \caption{\ac{SCRF} variational method}\label{alg:Reactionfieldvar}
  \begin{algorithmic}
    \STATE \underline{\textbf{Zeroth step:}}
    \STATE Do The iterative \ac{SCRF} $\rightarrow U_r^{(converged)}$
    \STATE $\gamma_s^{(n+1)} \leftarrow \gamma_s\big[(U_{vac}+ U_r^{(converged)}), \epsilon\big]$
    \STATE \underline{\textbf{n-th step:}}
    \STATE \textbf{On every SCF cycle do}
    \STATE $\gamma_s^{(n)} \leftarrow \gamma_s^{(n+1)}$
    \STATE $U_r^{(n+1)} \leftarrow \hat{\mathscr{P}} (\rho_{eff} + \gamma_s^{(n)})$
    \STATE $U_r^{(n)} \leftarrow U_r^{(n+1)}$
    \STATE $\gamma_s^{(n+1)} \leftarrow \gamma_s\big[(U_{vac}+ U_r^{(n)}), \epsilon\big]$
    \STATE Accelerator[$\gamma_s^{(n)}, \gamma_s^{(n+1)}] \rightarrow \gamma_s^{(n+1)\prime}$
    \STATE $\gamma_s^{(n+1)} \leftarrow\gamma_s^{(n+1)\prime}$
  \end{algorithmic}
\end{algorithm}
Where $\gamma_s^{(n+1)\prime}$ is an optimized $\gamma_s^{(n+1)}$ which is used
in the next iteration. The first Iterative \ac{SCRF} was done so that we would
start the Optimizing of the potential with an already good guess.
\section{Software used}
The problem was first implemented in VAMPYR \cite{Vampyr} which is a python interface to the
MRCPP \cite{MRCPP} code.  Then it was implemented in C++ using \mrchem \cite{MRchem}. All of the above softawares
are \ac{MW} software developed at Hylleraas centre of Quantum molecular sciences.
Both implementations are identical, except for slight changes for
performance improvement, such as using a KAIN accelerator,

\input{acronyms.tex}
\biblio
\end{document}

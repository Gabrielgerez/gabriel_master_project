\makeatletter
\def\input@path{{../}}
\makeatother
\documentclass[../master_thesis.tex]{subfiles}
\begin{document}
\chapter{Implementation}
\section{PCM}
\subsection{Dielectric function}
From \cite{FossoTande:2013ka} the cavity was implemented as follows

\begin{algorithm}[H]
  Cavity = $C$\;
  $r_i$ = position of the center of the sphere\;
  $R_i$ = radius of sphere\;
  $s(r) = |r - r_i| - R_i$\;
  $\sigma =$ width of slope \;
  \
 \For{each sphere $C_i$ at point r}{
  $C_i =\frac{1}{2}(1 - \erf(\frac{s(r)}{\sigma}))$\;
  $C = C \cdot C_i$\;
 }
\end{algorithm}
This would give us 1 on the inside of the Cavity and zero everywhere else. We
construct the dielectric function as a combination of the permitivity of free
space $\varepsilon_0 = 1.0$, the dielectric constant of the dielectric continuum
$\varepsilon_{\infty}$, and the Cavity in either a linear or exponential way \cite{FossoTande:2013ka} (Equations
\ref{eq:lineareps} and \ref{eq:exponentialeps} respectivelly). The linear
dielectric function is as follows:
\begin{equation}
  \varepsilon(r) = \varepsilon_0C(r) + \varepsilon_{\infty}(1 - C(r))\label{eq:lineareps}
\end{equation}

The exponential dielectric function is as follows:
\begin{equation}
  \varepsilon(r) = \varepsilon_0\exp(\log(\frac{\varepsilon_{\infty}}{\varepsilon_0})(1 - C(r)))\label{eq:exponentialeps}
\end{equation}

They were simply added to the end of the implementation of C so as to change the output of the cavity.

\subsection{Differences and limitiations in implementations}

\section{Reaction potential}

\section{Software used}
The problem was first implemented in Vampyr %cite and get the right font
which is a python %
Then it was implemented in C++ using MRchem %cite and get the right font
Both implementations are identical, except for slight changes for
performance improvement, such as a KAIN accelerator, %cite and correct font
iterating through $V_r$ instead of $V_{tot}$ % more explanation on this in previous or this section
and keeping the potential from cycle to cycle
in the c++ version of the model.


\begin{acronym}
\acro{AUS}[\href{https://www.sigma2.no/content/advanced-user-support}{AUS}]{Numerical Methods in Quantum Chemistry}
\acro{BO}{Born-Oppenheimer}
\acro{CTCC}[\href{http://www.ctcc.no}{CTCC}]{Centre for Theoretical and Computational Chemistry}
\acro{DC}{Dielectric Continuum}
\acro{DFT}{Density Functional Theory}
\acro{EFP}{Effective Fragment Potential}
\acro{EU}{European Union}
\acro{HF}{Hartree-Fock}
\acro{Hylleraas}[\href{https://www.mn.uio.no/hylleraas/english/}{Hylleraas}]{Hylleraas
  Centre for Quantum Molecular Sciences}
\acro{HPC}{High Performance Computing}
\acro{KTH}{Royal Institute of Technology}
\acro{LDA}{Local Density Approximation}
\acro{MCD}{Magnetic Circular Dichroism}
\acro{MCSCF}{Multiconfiguration Self Consistent Field}
\acro{MM}{Molecular Mechanics}
\acro{MW}{Multiwavelet}
\acro{NFR}{Norwegian Research Council}
\acro{NMQC}[\href{http://www.ctcc.no/events/conferences/2015/numeric-conference/}{NMQC}]{Numerical Methods in Quantum Chemistry}
\acro{NOTUR}[\href{https://www.notur.no/}{NOTUR}]{Norwegian Metacenter for Computational Science}
\acro{PCM}{Polarizable Continuum Model}
\acro{PI}{Primcipal Investigator}
\acro{QC}{Quantum Chemistry}
\acro{QM}{Quantum Mechanics}
\acro{QM/MM}{Quantum Mechanics/Molecular Mechanics}
\acro{ROA}{Raman Optical Activity}
\acro{SC}{semiconductor}
\acro{SCF}{Self Consistent Field}
\acro{SHG}{Second Harmonic Genertation}
\acro{STSM}{Short-term scientific mission}
\acro{TPA}{Two-Photon Absorption}
\acro{WP}{Work Package}
\acro{CBS}{Complete Basis Set}
\acro{TCG}{Theoretical Chemistry Group}
\acro{vdW}{van der Waals}
\acro{SE}{Schrödinger Equation}
\acro{PES}{Potential Energy Surface}
\acro{LCAO}{Linear Combination of Atomic Orbitals}
\acro{MRA}{Multi-Resolution Analysis}
\acro{NS}{Nonstandard}
\end{acronym}

\biblio
\end{document}

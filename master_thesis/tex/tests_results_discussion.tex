\makeatletter
\def\input@path{{../}}
\makeatother
\documentclass[../master_thesis.tex]{subfiles}
\begin{document}
\chapter{Results}\label{chap:Results}
\section{Overview of Tests}
We divided the tests of this implementation into four main types, (1) theoretical
correctness, (2) parametrization, (3) comparison and (4) tests of the variational implementation.
\begin{enumerate}

\item \textbf{Theoretical correctness:}
In the tests of theoretical correctness we test if our implementation gives the
results to problems as expected. The tests for this were comparisons to
the energy of  \ce{Li^+} in an environment with dielectric constant $\epsinf$
with the value we would get from the Born model. In the Born model the energy of a  %citation please
one atom ion in a solvent is the same as the energy of a point charge in the same solvent.
This energy is described as \cite{Tomasi:1994wt}
\begin{equation}\label{eq:bornenergy}
  E_{R}^{Born} =-\frac{\epsilon-1}{2 \epsilon} \frac{q^{2}}{R}
\end{equation}
A second comparison was done with Gauss' theorem\cite{Sorland} where we tested for the
following relation that should hold for a point charge
\begin{equation}\label{eq:Reactioncharge}
  \int \gamma_s = \frac{\epsinf - 1}{\epsinf} q
\end{equation}

\item \textbf{Parametrization}The parametrization tests were done by changing one parameter at a time while
comparing this same change to Gaussian calculations with different basis sets.
First we only checked the dependency on the radius and how changing it would
affect the reaction field energy with respect to Gaussian calculations of the same
radii. We then compared Gaussian calculations of Radius $R$ against \mrchem
calculations of Radius $R+0.2$ in an attempt to see an improvement in the
results.
A second parametrization test was done with only \ce{Li^+}
where we changed the relative precision of the \mrchem calculations to see how
they affect the energy with respect to the Gaussian energies. This was also done
on different radii.

\item \textbf{Comparison} We then took 4 molecules that were tested by Chipman in \cite{Chipman2002} and compared the
results from variation of radii against Gaussian results. The molecules used
were \ce{H_2O}, \ce{NO^+}, \ce{CN^-} and \ce{CH_3CONH_2}. These tests are classified into
spherical cavity tests and Molecular shape cavity tests. spherical cavity tests are performed in the
by varying the radius of the cavities for each molecule. The spherical cavity test was done for
all molecules except for \ce{CH_3CONH_2} as it was too big, which made calculations
in Gaussian and \mrchem extremely slow.

The Molecular shape cavity tests consist of making interlocking spheres centered on
each atom with radii equal to their center atom's Van der Waals radius. The Radii
used to represent the Van der Waals radii were the Bondi radii multiplied by a factor of
$1.2$ as outlined by \cite{Tomasi:1994wt}. These radii were later shifted by $0.2$ Bohr
so we were comparing \mrchem calculations of bigger radii to Gaussian calculations.

\item \textbf{Variational tests} As stated before in chapters \ref{chap:Solvent_effect} and \ref{chap:implementation}
a variational formulation of the Reaction field Problem was implemented in this thesis,
The tables in Appendix \ref{Datatables} show a row which is labeled "Variational",
though the variational implementation behaved in a irregular way. We will show
the Reaction energy plots for the variational implementation for both water
and \ce{Li^+} as those are the ones that had the most data points. We leave the
reader to evaluate the variational energies for the other molecules.

\end{enumerate}

\section{Data}
\subsection{Theoretical correctness Tests}
We calculated the integral on the left hand side of Equation \ref{eq:Reactioncharge} for a point charge
with charge $q=3$ and compared it to the exact value calculated as shown in the right hand side of the same
equation. This is shown in Tables \ref{tab:Intgamma2} and \ref{tab:intgamma80}.

We then evaluated the Reaction field energy of the same point charge and compared it
to an exact value calculated as in Equation  \ref{eq:bornenergy}. Tables \ref{tab:Er2}
and \ref{tab:Er80} show this.

For both the tests above we calculated the values with two Radii, $R = 3.0,\ 4.0$ Bohr,
two sets of Dielectric constants, $\epsinf = 2,\ 80$, two transition widths, $\sigma = 0.1,\ 0.2$,
and two values for the relative precision, $1e-4,\ 1e-6$.

\begin{table}[!htbp]
\caption{Reaction charge for a point charge of $q = 3$ and $\epsinf = 2$ calculated with differing precision, transition width ($\sigma$) and cavity radius (Bohr) compared to the exact values}
\resizebox{\textwidth}{!}{
\begin{tabular}{l|r|r|r|r|r}

Radius & \multicolumn{1}{l|}{Prec.} & $\sigma = $0.2 & \multicolumn{1}{l|}{Rel. Diff.} & $\sigma = $0.1 & \multicolumn{1}{l|}{Rel. Diff.} \\ \hline
\multicolumn{ 1}{r|}{3.0} & 1E-04 & -1.499964 & -2.41E-05 & -1.504559 & 3.04E-03 \\
\multicolumn{ 1}{l|}{} & 1E-06 & -1.499999 & -7.43E-07 & -1.500001 & 4.17E-07 \\ \hline
\multicolumn{ 1}{r|}{4.0} & 1E-04 & -1.499602 & -2.66E-04 & -1.503913 & 2.61E-03 \\
\multicolumn{ 1}{l|}{} & 1E-06 & -1.500002 & 1.21E-06 & -1.499659 & -2.27E-04 \\ \hline
\multicolumn{ 1}{|l|}{Exact} & \multicolumn{1}{|l|}{} & -1.500000 & \multicolumn{1}{l|}{} & -1.500000 & \multicolumn{1}{l|}{} \\ \hline
\end{tabular}}
\label{tab:Intgamma2}
\end{table}

\begin{table}[!htbp]
  \caption{Reaction charge for a point charge of $q = 3$ and $\epsinf = 80$ calculated with differing precision, transition width ($\sigma$) and cavity radius (Bohr) compared to the exact values}
  \resizebox{\textwidth}{!}{
  \begin{tabular}{l|r|r|r|r|r}
    Radius & \multicolumn{1}{l|}{Prec.} & $\sigma = $0.2 & \multicolumn{1}{l|}{Rel. Diff.} & $\sigma = $0.1 & \multicolumn{1}{l}{Rel. Diff.} \\ \hline
    \multicolumn{ 1}{r|}{3.0} & 1.00E-04 & -2.96495 & 8.26E-04 & -2.96167 & -2.81E-04 \\
    \multicolumn{ 1}{l|}{} & 1.00E-06 & -2.96250 & 2.92E-07 & -2.96235 & -5.16E-05 \\ \hline
    \multicolumn{ 1}{r|}{4.0} & 1.00E-04 & -2.96400 & 5.06E-04 & -2.95605 & -2.18E-03 \\
    \multicolumn{ 1}{l|}{} & 1.00E-06 & -2.96243 & -2.20E-05 & -2.96215 & -1.17E-04 \\ \hline
    \multicolumn{ 1}{|l|}{exact} & \multicolumn{1}{|l|}{} & -2.96250 & \multicolumn{1}{l|}{} & -2.96250 & \multicolumn{1}{l|}{} \\ \hline
  \end{tabular}}
  \label{tab:intgamma80}
\end{table}

\begin{table}[!htbp]
\caption{Reaction field energy for a point charge of $q = 3$ and $\epsinf = 2$ calculated with differing precision, transition width ($\sigma$) and cavity radius (Bohr) compared to the exact values}
\resizebox{\textwidth}{!}{
\begin{tabular}{l|l|r|r|r|r|r}
Radius & Exact & \multicolumn{1}{l|}{Prec.} & $\sigma = $0.2 & \multicolumn{1}{l|}{Rel. Diff.} & $\sigma = $0.1 & \multicolumn{1}{l|}{Rel. Diff.} \\ \hline
\multicolumn{1}{r|}{3.0} & \multicolumn{ 1}{r|}{-0.7500} & \multicolumn{ 1}{r|}{1E-04} & -0.7586 & 1.14E-02 & -0.7560 & 7.96E-03 \\
 & \multicolumn{ 1}{l|}{} & \multicolumn{ 1}{r|}{1E-06} & -0.7586 & 1.15E-02 & -0.7539 & 5.15E-03 \\ \hline
\multicolumn{1}{r|}{4.0} & \multicolumn{ 1}{r|}{-0.5625} & \multicolumn{ 1}{r|}{1E-04} & -0.5670 & 7.95E-03 & -0.5660 & 6.27E-03 \\
 & \multicolumn{ 1}{|l|}{} & \multicolumn{ 1}{r|}{1E-06} & -0.5671 & 8.17E-03 & -0.5645 & 3.53E-03 \\
\end{tabular}}
\label{tab:Er2}
\end{table}

\begin{table}[!htbp]
\caption{Reaction field energy for a point charge of $q = 3$ and $\epsinf = 80$ calculated with differing precision, transition width ($\sigma$) and cavity radius (Bohr) compared to the exact values}
\resizebox{\textwidth}{!}{
\begin{tabular}{l|l|r|r|r|r|r}
Radius & Exact & \multicolumn{1}{l|}{Prec.} & $\sigma = $0.2 & \multicolumn{1}{l|}{Rel. Diff.} & $\sigma = $0.1 & \multicolumn{1}{l|}{Rel. Diff.} \\ \hline
\multicolumn{1}{r|}{3} & \multicolumn{ 1}{r|}{-1.48125} & 1.00E-04 & -1.55781 & 5.17E-02 & -1.51700 & 2.41E-02 \\
 & \multicolumn{ 1}{l|}{} & 1.00E-06 & -1.55658 & 5.09E-02 & -1.51731 & 2.43E-02 \\ \hline
\multicolumn{1}{r|}{4} & \multicolumn{ 1}{r|}{-1.1109375} & 1.00E-04 & -1.15306 & 3.79E-02 & -1.12961 & 1.68E-02 \\
 & \multicolumn{ 1}{l|}{} & 1.00E-06 & -1.15241 & 3.73E-02 & -1.13093 & 1.80E-02 \\
\end{tabular}}
\label{tab:Er80}
\end{table}

\clearpage


\subsection{Parametrization Tests}\label{sec:paratests}
The data tables containing all results can be found in appendix \ref{Datatables}, following
tables will show a small sample so the reader can make have a understanding of
the tables shown there.
We first varied the Radius of the cavity for water and lithium. The following
table \ref{tab:rawwaterdata}  presents the data for the energy
calculations of \ce{H_2O} with the three first cavity radii as an example of the layout of the
tables.
These are the total  energy of the system including the solvent effect contributions.
The same type of tables were used for the rest of the systems.
\begin{table}[!htbp]
  \caption{Total Energy Calculations example for Water in Water. Energy in Hartree and radii of the cavity in Bohr}
  \begin{center}
    \begin{tabular}{l|r|r|r}
      Basis & $R =3.6$ & $R=3.7$ & $R=3.8$ \\  \hline
      Cc-pVDZ & -7.6039e+01 & -7.6038e+01 & -7.6036e+01 \\
      Cc-pVTZ & -7.6070e+01 & -7.6069e+01 & -7.6067e+01 \\
      Cc-pVQZ & -7.6078e+01 & -7.6076e+01 & -7.6075e+01 \\
      Cc-pV5Z & -7.6080e+01 & -7.6079e+01 & -7.6077e+01 \\
      Aug-cc-pVDZ & -7.6054e+01 & -7.6053e+01 & -7.6052e+01 \\
      Aug-cc-pVTZ & -7.6074e+01 & -7.6072e+01 & -7.6071e+01 \\
      Aug-cc-pVQZ & -7.6079e+01 & -7.6077e+01 & -7.6076e+01 \\
      Aug-cc-pV5Z & -7.6080e+01 & -7.6079e+01 & -7.6077e+01 \\
      daug-cc-pVDZ & -7.6055e+01 & -7.6053e+01 & -7.6052e+01 \\
      daug-cc-pVTZ & -7.6074e+01 & -7.6072e+01 & -7.6071e+01 \\
      daug-cc-pVQZ & -7.6079e+01 & -7.6077e+01 & -7.6076e+01 \\
      daug-cc-pV5Z & -7.6080e+01 & -7.6079e+01 & -7.6077e+01 \\
      \mrchem & -7.6085E+01 & -7.6083E+01 & -7.6081E+01 \\
    \end{tabular}
  \end{center}
  \label{tab:rawwaterdata}
\end{table}

To Calculate the Reaction field energy we took a gas phase calculation
of a basis set and subtracted it from the total energy calculated with the same
basis set. In \mrchem this was done using the same relative precision for both
the gas phase and the solvent calculations. The following equation was used to
calculate the Reaction field Energy $E_r$
\begin{equation}\label{eq:deltaer}
  E_r = E_{tot} - E_{vac}
\end{equation}
Examples of $E_r$ for the first three cavity radii for water obtained from the operation
in Equation \ref{eq:deltaer} can be seen in table \ref{tab:Erwatdata}

\begin{table}[!htbp]
\caption{Reaction Field Energy Calculations example for Water in Water. Energy in Hartree and radii of the cavity in Bohr}
\begin{center}
\begin{tabular}{l|r|r|r|r}
Basis & 3.6 & 3.7 & 3.8 \\\hline
Cc-pVDZ & -1.2450E-02 & -1.0998E-02 & -9.7804E-03 \\
Cc-pVTZ & -1.3097E-02 & -1.1545E-02 & -1.0243E-02 \\
Cc-pVQZ & -1.3218E-02 & -1.1651E-02 & -1.0334E-02 \\
Cc-pV5Z & -1.3284E-02 & -1.1713E-02 & -1.0393E-02 \\
Aug-cc-pVDZ & -1.3190E-02 & -1.1634E-02 & -1.0328E-02 \\
Aug-cc-pVTZ & -1.3238E-02 & -1.1670E-02 & -1.0353E-02 \\
Aug-cc-pVQZ & -1.3221E-02 & -1.1655E-02 & -1.0338E-02 \\
Aug-cc-pV5Z & -1.3223E-02 & -1.1655E-02 & -1.0337E-02 \\
daug-cc-pVDZ & -1.3228E-02 & -1.1665E-02 & -1.0351E-02 \\
daug-cc-pVTZ & -1.3243E-02 & -1.1675E-02 & -1.0357E-02 \\
daug-cc-pVQZ & -1.3223E-02 & -1.1656E-02 & -1.0340E-02 \\
daug-cc-pV5Z & -1.3224E-02 & -1.1655E-02 & -1.0337E-02 \\
mrchem & -1.8036E-02 & -1.5494E-02 & -1.3437E-02 \\
\end{tabular}
\end{center}
\label{tab:Erwatdata}
\end{table}


We now plot the data from the total energy and reaction energy tables for
both water and \ce{Li^+}.

The plots for the Reaction energy of water for
both the Gaussian and  \mrchem calculations can be seen in Figure \ref{fig:watEnergyplotsdaug}.
The same type of plots for \ce{Li^+} can be seen in Figure \ref{fig:lipEnergyplotsdaug}.
In both of these figure we are comparing the energy from \mrchem to sets of four
curves formed each of double, triple, quadruple and quintuple zeta Dunning's correlation
consistent \cite{doi:10.1063/1.456153} basis sets as implemented in Gaussian \cite{G09}.

While in this chapter we only look at double augmented basis, in Appendix \ref{Figures}
the reader can see this same comparison for augmented and standard basis sets for
both these tests and subsequent ones.

The \mrchem energy values $E_{MRChem}$ for each radii were compared to the
corresponding values of each of the different basis set calculations in
Gaussian  $E_{Gaussian}^{basis}$ by finding the relative difference $d_r$
between them as
\begin{equation}\label{eq:reldiff}
  d_r = \frac{E_{Gaussian}^{basis} - E_{MRChem}}{E_{MRChem}}
\end{equation}
The operation in Equation \ref{eq:reldiff} was applied to both water and \ce{Li^+} for all the
substrate molecules, giving the following figures \ref{fig:watreldiffdaug} and \ref{fig:lipreldiffdaug}.

We then shifted the cavity radius of the \mrchem calculations for both water and
\ce{Li^+} so they were $0.2$ Bohr bigger and compared them to Gaussian calculations
with an unshifted Radius. The relative Difference plots for water and \ce{Li^+} can be seen
in Figures \ref{fig:watreldiff02daug} and \ref{fig:lipreldiff02daug} respectively.

Lastly we computed the reaction field energies of \ce{Li^+} with four different
relative precision; $1e-3, 1e-4, 1e-5,\  \text{and}\  1e-6$. We compared these
to the reaction field energies calculated with the most complete basis set:
\verb!daug-cc-pV5Z! as described in
\begin{equation}\label{eq:difgauss}
  d_r = \frac{E_{Li^+} - E_{Gaussian}}{E_{Gaussian}}
\end{equation}
Figure \ref{fig:lipprecreldef} shows the plots of the operation in Equation \ref{eq:difgauss} above.
First all of the relative differences plotted together, then the second one contains
only the results for relative precision $1e(-4), 1e(-5),\  \text{and}\  1e(-6)$ since
the results with relative precision $1e(-3)$ are over five times larger than the other ones.

\begin{figure}[!htb]
  \centering
    \includegraphics[width=0.75\linewidth]{img/Erdaugwat.png}
  \caption{Reaction field energy of Water in a water solution, calculated with relative precision $e-05$ in \mrchem and with double augmented basis sets in Gaussian}
  \label{fig:watEnergyplotsdaug}
\end{figure}

\begin{figure}[!htb]
  \centering
    \includegraphics[width=0.75\linewidth]{img/Erdauglip.png}
  \caption{Reaction field energy of \ce{Li^+} in a water solution, calculated with relative precision $e-05$ in \mrchem  and with double augmented basis sets in Gaussian}
  \label{fig:lipEnergyplotsdaug}
\end{figure}



\begin{figure}[!htb]
  \centering
    \includegraphics[width=\linewidth]{img/watdaugreldiff.png}
  \caption{Relative difference between the Reaction field energy of Water in a water solution calculated with with relative precision $e-05$ in \mrchem
   and with double augmented basis sets in Gaussian}
  \label{fig:watreldiffdaug}
\end{figure}

\begin{figure}[!htb]
  \centering
  \includegraphics[width=\linewidth]{img/lipdaugreldiff.png}
  \caption{Relative difference between the Reaction field energy of\ce{Li^+} in a water solution calculated with relative precision $e-05$ in \mrchem
  and with double augmented basis sets in Gaussian}
  \label{fig:lipreldiffdaug}
\end{figure}



\begin{figure}[!htb]
  \centering
    \includegraphics[width=\linewidth]{img/watdaugreldiff02.png}
  \caption{Relative difference between the Reaction field energy of Water in a water solution calculated with with relative precision $e-05$ in \mrchem
  and with double augmented basis sets in Gaussian}
  \label{fig:watreldiff02daug}
\end{figure}

\begin{figure}[!htb]
  \centering
    \includegraphics[width=\linewidth]{img/lipdaugreldiff02.png}
  \caption{Relative difference between the Reaction field energy of \ce{Li^+} in a water solution calculated with relative precision $e-05$ in \mrchem
  and with double augmented basis sets in Gaussian}
  \label{fig:lipreldiff02daug}
\end{figure}



\begin{figure}[!htb]
  \centering
  \begin{subfigure}[b]{0.75\linewidth}
    \includegraphics[width=\linewidth]{img/lipprecallreldiff.png}
  \end{subfigure}
  \begin{subfigure}[b]{0.75\linewidth}
    \includegraphics[width=\linewidth]{img/lipprecallreldiffexcl.png}
  \end{subfigure}
  \caption{Relative difference between the Reaction field energy of \ce{Li^+} in a water solution calculated with different relative precisions in \mrchem  and same calculations in Gaussian with daug-cc-pV5Z}
  \label{fig:lipprecreldef}
\end{figure}
\clearpage

\subsection{Comparison Tests}
The substrates studied in these tests are the same 4 substrates studied by Chipman in \cite{Chipman2002},
namely, water, \ce{NO^+}, \ce{CN^-} and \ce{CH_3CONH_2}. Two sets of comparison tests
were performed. Firstly, spherical cavity calculations where for each calculation the radius of the
cavity was varied. Secondly, molecular shape cavities calculations were performed on
all the substrates. Of the second type of test only two sets of radii were calculated, for
each substrate, cavities based on the Bondi radii multiplied by $1.2$, and the same cavities
but their radii were increased by $0.2$Bohr.


The results for the spherical cavity tests for water can be seen in figures
\ref{fig:watEnergyplotsdaug} and \ref{fig:watreldiffdaug}, and the results for the
molecular shape tests for water can be seen on table \ref{tab:watabcreldiff}.
For \ce{NO^+} the plots for the single sphere cavity tests can be seen in figures
\ref{fig:nopEnergyplotsdaug} and \ref{fig:nopreldiffdaug} and the molecular
shape cavity test result can be seen in table \ref{tab:nopabcreldiff}.
For \ce{CN^+} one can see the plots for the single sphere cavity on figures
\ref{fig:cyanEnergyplotsdaug} and \ref{fig:cyanreldiffdaug} while the
molecular shape results are on table \ref{tab:cyanabcreldiff}. The molecular shape
test for the incremented radii did not converge for \ce{CN^-}, but comparisons between
the standard scaled Bondi radii and the single sphere cavity can still be compared.

Some of the data points for \ce{CN^-} and \ce{NO^+} are missing. The reason for this
is that the calculations for these values either never completed or gave extreme
outliers that made it harder to see the trend of the correctly converged values.
All of the available values, those which completed, can be seen in their correspnding
tables in appendix \ref{Datatables}.

There are no single cavity tests for \ce{CH_3CONH_2} due to the size of the molecule
making single sphere calculations extremely slow for both Gaussian and \mrchem.
The only tests that were ran where molecular shape tests. Gaussian did not converge
after 72 hours of running with \verb!daug-cc-pV5Z!, therefore, the comparison values
are missing from the tables. The \ce{CH_3CONH_2} results are presented in table
\ref{tab:acetamidabcreldiff}.

The molecular shape cavity results shown are just the relative difference between
\mrchem and Gaussian values as shown in Equation \ref{eq:reldiff}. For the rest of the
tables see appendix \ref{Datatables}.

\begin{figure}[!htb]
  \centering
    \includegraphics[width=\linewidth]{img/Erdaugnop.png}
  \caption[Energy plots for \ce{NO^+}]{Reaction field energy of \ce{NO^+} in a water solution, calculated with \mrchem
  and with double augmented basis sets in Gaussian}
  \label{fig:nopEnergyplotsdaug}
\end{figure}

\begin{figure}[!htb]
  \centering
    \includegraphics[width=\linewidth]{img/Erdaugcyan.png}
  \caption{Reaction field energy of \ce{CN^-} in a water solution, calculated with \mrchem
  and with different basis sets in Gaussian}
  \label{fig:cyanEnergyplotsdaug}
\end{figure}

\begin{figure}[!htb]
  \centering
    \includegraphics[width=\linewidth]{img/nopdaugreldiff.png}
    \caption{Relative difference between the Reaction field energy of \ce{NO^+} in a water solution calculated with \mrchem
  and with different basis sets in Gaussian}
  \label{fig:nopreldiffdaug}
\end{figure}

\begin{figure}[!htb]
  \centering
    \includegraphics[width=\linewidth]{img/cyandaugreldiff.png}
  \caption{Relative difference between the Reaction field energy of \ce{CN^-} in a water solution calculated with \mrchem
  and with double augmented basis sets in Gaussian}
  \label{fig:cyanreldiffdaug}
\end{figure}


\begin{table}[htbp]
\caption{Relative difference between Gaussian and \mrchem results from Molecular shape cavity  test for water}
\begin{tabular}{l|r|r}
Basis & \multicolumn{1}{l|}{Van der Waals Radii} & \multicolumn{1}{l|}{Van der Waals Radii $+ 0.2$ Bohr} \\ \hline
Cc-pVDZ & -0.150906543613473 & 0.04037455582641 \\
Cc-pVTZ & -0.128295351717092 & 0.068079525814065 \\
Cc-pVQZ & -0.128952645914318 & 0.067274158450387 \\
Cc-pV5Z & -0.129254059406514 & 0.06690484347604 \\
Aug-cc-pVDZ & -0.129130316675487 & 0.067056462579757 \\
Aug-cc-pVTZ & -0.13428775618655 & 0.060737171340182 \\
Aug-cc-pVQZ & -0.136838481699814 & 0.057611826417386 \\
Aug-cc-pV5Z & -0.136314953280292 & 0.058253293670285 \\
daug-cc-pVDZ & -0.128163492341192 & 0.068241090054846 \\
daug-cc-pVTZ & -0.133787441562786 & 0.061350195266532 \\
daug-cc-pVQZ & -0.136489745183211 & 0.058039125198024 \\
daug-cc-pV5Z & -0.136235086647359 & 0.058351152406726 \\
\end{tabular}
\label{tab:watabcreldiff}
\end{table}


\begin{table}[htbp]
\caption{Relative difference between Gaussian and \mrchem results from Molecular shape cavity  test for \ce{NO^+}}
\begin{tabular}{l|r|r}
Basis & \multicolumn{1}{l|}{Van der Waals Radii} & \multicolumn{1}{l|}{Van der Waals Radii $+ 0.2$ Bohr} \\ \hline
Cc-pVDZ & -0.03598423 & 0.02044325 \\
Cc-pVTZ & -0.03692221 & 0.01945037 \\
Cc-pVQZ & -0.03747002 & 0.01887050 \\
Cc-pV5Z & -0.03810456 & 0.01819881 \\
Aug-cc-pVDZ & -0.03906862 & 0.01717832 \\
Aug-cc-pVTZ & -0.03801963 & 0.01828871 \\
Aug-cc-pVQZ & -0.03792710 & 0.01838665 \\
Aug-cc-pV5Z & -0.03814585 & 0.01815510 \\
daug-cc-pVDZ & -0.03897379 & 0.01727870 \\
daug-cc-pVTZ & -0.03811073 & 0.01819228 \\
daug-cc-pVQZ & -0.03799786 & 0.01831176 \\
daug-cc-pV5Z & -0.03813936 & 0.01816197 \\
\end{tabular}
\label{tab:nopabcreldiff}
\end{table}


\begin{table}[!htbp]
\caption{Relative difference between Gaussian and \mrchem results from Molecular shape cavity  test for \ce{CN^-}}
\begin{tabular}{l|r}
Basis & \multicolumn{1}{l|}{Van der Waals Radii} \\ \hline
Cc-pVDZ & 0.01805797 \\
Cc-pVTZ & 0.00402722 \\
Cc-pVQZ & -0.00689520 \\
Cc-pV5Z & -0.01760704 \\
Aug-cc-pVDZ & -0.02534307 \\
Aug-cc-pVTZ & -0.02484616 \\
Aug-cc-pVQZ & -0.02473155 \\
Aug-cc-pV5Z & -0.02453156 \\
daug-cc-pVDZ & -0.02428967 \\
daug-cc-pVTZ & -0.02475331 \\
daug-cc-pVQZ & -0.02472359 \\
daug-cc-pV5Z & -0.02449823 \\
\end{tabular}
\label{tab:cyanabcreldiff}
\end{table}

\begin{table}[htbp]
\caption{Relative difference between Gaussian and \mrchem results from Molecular shape cavity  test for \ce{CH_3CONH_2}}
\begin{tabular}{l|r|r}
Basis & \multicolumn{1}{l|}{Van der Waals Radii} & \multicolumn{1}{l|}{Van der Waals Radii$+0.2$ Bohr} \\ \hline
Cc-pVDZ & -0.186177328600554 & -0.025757341450533 \\
Cc-pVTZ & -0.13954657945224 & 0.030065218693051 \\
Cc-pVQZ & -0.12431319516031 & 0.048301393885559 \\
Cc-pV5Z & -0.12131273376434 & 0.051893303511593 \\
Aug-cc-pVDZ & -0.113966441504817 & 0.060687690240943 \\
Aug-cc-pVTZ & -0.120388767462925 & 0.052999401212647 \\
Aug-cc-pVQZ & -0.122585402181545 & 0.050369767849829 \\
Aug-cc-pV5Z & -0.122435027608944 & 0.050549784121847 \\
daug-cc-pVDZ & -0.117198265984418 & 0.056818811449993 \\
daug-cc-pVTZ & -0.121288328616877 & 0.051922519379598 \\
daug-cc-pVQZ & -0.122478065257119 & 0.050498262931483 \\
\end{tabular}
\label{tab:acetamidabcreldiff}
\end{table}
\clearpage


\subsection{Variational implementation Tests}
Figures \ref{fig:watvarEr} and \ref{fig:lipvarEr} show the reaction field energy
calculated with the variational implementation for water and \ce{Li^+}
respectively. Figures \ref{fig:watreldiffvardaug} and \ref{fig:lipreldiffvardaug} show the
relative difference as calculated with Equation \ref{eq:reldiff} for water and \ce{Li^+}
respectively.

\begin{figure}[!htb]
  \centering
  \includegraphics[width=0.75\linewidth]{img/watvarEr.png}
  \caption{Relative difference between the Reaction field energy of \ce{CN^-} in a water solution calculated with \mrchem
  and with different basis sets in Gaussian}
  \label{fig:watvarEr}
\end{figure}

\begin{figure}[!htb]
  \centering
  \includegraphics[width=0.75\linewidth]{img/lipvarEr.png}
  \caption{Relative difference between the Reaction field energy of \ce{CN^-} in a water solution calculated with \mrchem
  and with different basis sets in Gaussian}
  \label{fig:lipvarEr}
\end{figure}

\begin{figure}[!htb]
  \centering
    \includegraphics[width=\linewidth]{img/watitervarreldiff.png}
  \caption{Relative difference between the Reaction field energy of \ce{CN^-} in a water solution calculated with \mrchem
  and with different basis sets in Gaussian}
  \label{fig:watreldiffvardaug}
\end{figure}

\begin{figure}[!htb]
  \centering
    \includegraphics[width=\linewidth]{img/lipitervarreldiff.png}
  \caption{Relative difference between the Reaction field energy of \ce{CN^-} in a water solution calculated with \mrchem
  and with different basis sets in Gaussian}
  \label{fig:lipreldiffvardaug}
\end{figure}
\clearpage


\section{Discussion}
\subsection{Theoretical correctness Discussion}
The first thing to mention about the results from tables \ref{tab:Intgamma2}, \ref{tab:Er2},
\ref{tab:intgamma80} and \ref{tab:Er80} is that the calculated values are quite close
to the exact values. A reason for them not being entirely identical might be that while the
exact values are calculated by assuming we have a point charge, the \mrchem values
are calculated with densities representing the nucleus of \ce{Li^+} without the
electron density.

We can see that, for both the energy calculations and for the integral for both of the dielectric values,
increasing the tightness of the precision, from $1e-4 \text{ to } 1e-6$ improves the relative
difference for all the values, except for the energy from width $0.2$ and dielectric constant $80$, where
it worsens by $ 0.01\% $. This is most likely caused by escaped density, since higher precision makes
it possible for the density tails to extend further in the projection.

Increasing the radius does not improve the differences of the integrals, except for
$\epsinf = 2 \text{ with } \sigma = 0.1 \text{ and } prec. = 1e-4$, and for
$\epsinf = 80 \text{ with } \sigma = 0.2 \text{ and } prec. = 1e-4$ where it improves by
$0.1\% \text{ for } \epsinf = 2$ and $0.03\%$ for $\epsinf = 80$. The cause for
this decrease in the quality of the \mrchem results might be because of the surface
charge effect being diminished by the bigger cavity. The reason it gets better at
only two sets of parameters might be caused by numerical noise, given the size of
the variation. It might also be because of the escaped density being more easily
negligible by those sets of parameters.

Increasing the radius of the cavity and making the transition steeper affects
positively the energy results for both $\epsinf$ values. This is probably because
more of the density is taken into account when calculating the energy.

The relative differences for the $\epsinf = 80$ are not as good as for $\epsinf = 2$.
The most likely cause is the transition becomes steeper relative
to the width and Radius parameters due to the high difference between $\epsinf \text{ and } \epso$.
This leads to \mrchem having difficulties projecting the cavity, which in turn,
increase the numerical noise in the calculation.

\subsection{Parametrization Discussion}
The values for water reaction energy with a cavity radius of $5$ Bohr diverged, therefore
we removed them in order to better visualize the trends.
We can see in figures \ref{fig:watreldiffdaug} and \ref{fig:lipreldiffdaug}
that the relative difference decays, diminishes to a stable value, the bigger the cavity is. The method Gaussian
uses to calculate the reaction field describes the transition from inside the
cavity to the outside as happening at a boundary, this being the surface of the cavity.
The problem is solved at the surface of the cavity, which is assumed to be two dimensional,
so the transition is noncontinuous, as explained in chapter \ref{chap:Solvent_effect}.
Our cavity is defined as an analytical transition with a parameter $\sigma$ which
controls the width of the the cavity surface, so that we get a smooth transition
between the inside and the outside of the cavity.

Two main factors are responsible for the decay in relative difference that we observe.
Firstly the bigger the radius, the smaller the difference is relative to the
size of the cavity. This means that at larger distances our cavity resembles
more and more that of a discontinuous transition such as the ones used by
Gaussian. The second factor is the amount of the charge density that is contained within
the cavity. The effect of this can be seen on the tests that followed.

As we can see on the figures \ref{fig:watreldiff02daug} and \ref{fig:lipreldiff02daug}
there is a significant improvement on the relative difference when we compare with
bigger radii of \mrchem calculations to smaller radii of Gaussian calculations.
This is caused by the smooth cavity. Since the transition is wider in our implementation
there is more density that is situated outside the cavity than in cavities of same radius
from Gaussian calculations. This was accounted when we increased the radii, and we still
see the same decay in difference that we saw in the standard comparison.

The graphs from Figure \ref{fig:lipprecreldef} tell us two things about our results:
Firstly at precision lower than $1e-03$ we get results that are at least $20\%$ different
than the best Gaussian results we could run for small radii and diverge from the
Gaussian results for bigger radii. Secondly we see that we don't need a relatively big
precision to get stable results which are consistent with greater previsions and
that also approach the Gaussian results at bigger radii.

On the other hand, it is hard to decide if this is applicable to bigger systems,
as the Lithium cation is fairly simple, whereas
other systems might not be so simple.

\subsection{Comparison Discussion}
Same as with the tests for \ce{Li^+} and water, the Figures \ref{fig:nopreldiffdaug}
and  \ref{fig:cyanreldiffdaug} show that both relative differences for \ce{CN^-}
and \ce{NO^+} decay at higher radii.

The relative differences for the results for \ce{NO^+}  seem to decay faster
than the other molecules, but it
is hard to see due to the lack of data points. Assuming that it decays faster,
one can explain this as being due to the fact that the electron density is
closer to the atoms, making it easier for the cavity to contain most of it.
The opposite can be seen for \ce{CN^-}. The explanation for this might go through
the same line of thought as that for \ce{NO^+}, but now since the molecule is negatively
charged, the density is more diffuse, making it harder to contain it in its entirety.

The divergence around at interval $(4.0, 4.7)$ can be attributed to the way the
functions are projected. The cavity function is very sharp at the transition. In order to
project it correctly one needs to go to very low scales.
Additionally, the intervals are
powers of two, meaning that there is always the discontinuity from the wavelets at the
$4.0$ point, which is hard to correct. This discontinuity brings forth numerical
noise that gets amplified when the derivative of the function is calculated to solve the
\ac{GPE}.
These errors occur in this implementation as it is still in development. These
problems are expected to be diminished or removed in future revisions. One way we
could attempt to fix is by defining a better derivative for the Cavity.
Fosso--Tande defined the derivative of the Cavity as combinations of analytical
derivatives \cite{FossoTande:2013ka}. This definition supposedly would help
remove numerical noise at the transition, since one can create these analytical
derivatives before projecting them into the \ac{MW} basis.

The tables for the Molecular shape cavity tests  show us that for most the molecules, the
relative difference doesn't vary too much when going to Molecular shape cavity or that it gets
much better, as is the case for \ce{NO^+} in Table \ref{tab:nopabcreldiff}.
It can be seen that, while its spherical cavity plot, Figure \ref{fig:nopreldiffdaug}, does not go
lower than a relative difference of $10\%$, its Molecular shape cavity tests have a relative difference
of less than $4\%$. The same can be said for the results for \ce{CN^-}, in Table \ref{tab:cyanabcreldiff},
which got to less than $3\%$ from the Gaussian results, while it never approached $10\%$
with spherical cavity, Figure \ref{fig:cyanreldiffdaug}.

The water results, in table \ref{tab:watabcreldiff} do not go too far from the best relative difference
shown in Figure \ref{fig:watreldiffdaug}. They do significantly better when increasing
the Radius of the interlocking spheres, a decrease in the relative difference of
around $9-6\%$. This decrease is seen in the other molecules as well, where,
for \ce{NO^+}, the decrease is around $1-2\%$.

\ce{CH_3CONH_2}, in Table \ref{tab:acetamidabcreldiff} has a difference of at
best $11.4\%$ and at worst $18.6\%$. Increasing the radii of the interlocking spheres
has the same effect that was observed for water, but stronger. Here they decrease by
$ 16-5\%$.  This tells us that in \mrchem, in order to get relatively equivalent results to
Gaussian, one need to define a cavity that is a bit bigger than the one used in Gaussian.
This makes sense as the cavity surface, because of the width, starts earlier for us
than it does for Gaussian. This leads to more of the density escaping. Increasing the
radius gives us a cavity that is effectively the same as Gaussian's.

\subsection{Variational implementation discussion}
The results for the variational implementation of the \ac{GPE} for water and \ce{Li^+}
is shown in Figures \ref{fig:watreldiffvardaug} and \ref{fig:lipreldiffvardaug}
respectively.

Ideally, the results from the iterative and variational implementation should be
almost equal, with differences between them being for the most part caused by numerical
noise. This is because we are using the same \ac{GPE}, but different implementation,
for both of them.

\ce{Li^+} behaves almost as expected, with very small differences between implementations,
which can be explained as numerical noise. Water on the other hand shows signs that,
for bigger systems, the errors are not numerical noise only. The water differences imply
that there is an error in the implementation, which gets augmented for bigger systems.
This is further strengthened by the fact that the variational values for the other
molecules had the same trend, as can be seen in the Tables of appendix \ref{Datatables}.

Something worth noting is that the values appear to behave systematically,
which strengthens the possibility of something being implemented wrongly, and most
importantly, that it can be accounted for and corrected in future revisions of
this implementation.

\section{Concluding remarks}
The Theoretical correctness tests showed that the implementation is, at its base,
theoretically correct.

We saw that when increasing the radii of the cavity in calculations, the difference
between the Gaussian calculations and the \mrchem calculations decayed, as is
expected.
Having a tighter precision than $1e-4$ helps to have better results, but is not entirely
necessary, as the energy values at better precision values are marginally better.

We see that all the solutes used in \cite{Chipman2002} have relative differences that
decay with bigger radii. We also saw that using bigger \mrchem cavities where Gaussian
used smaller gave us better relative differences in molecular shape Cavities.
The \mrchem calculations did not converge for some values of \ce{CN^+} and \ce{NO^+}
which is most likely caused by instabilities in the derivatives of the cavity, which
can be improved by following Fosso--Tande \cite{FossoTande:2013ka}.

The variational implementation gives acceptable results for small systems such
as \ce{Li^+} but for bigger systems, it shows clear signs of errors in the implementation.
The variational implementation is still in its early days when it comes to development,
and the results still behave systematically, which tells us that they can be accounted for and
improved upon.

\section{Areas of improvement/future development}
Given that the iterative values gave results as expected we can say that the
implementation is correct. An improvement would be to try to change the cavity
definition so as to take into account more of the electron density. One way that is
possible right now is to decrease the width of the transition, another is to simply use
bigger cavities as these are shown to give better results. Both of those are possible
with the implementation used now, but decreasing the width of the transition might
make the cavity harder to project.

A better solution to this is to implement the derivative of the caivty as
a combination of analytical derivatives, as Fosso--Tande did in \cite{FossoTande:2013ka}, which
might let us get better results with the same parameters used in the tests of this chapter.

The goal in the future is to apply these solutions and test if they better the
results of the variational method. After that we will start seeing how molecular properties
are affected by this.


\input{acronyms.tex}
\biblio
\end{document}

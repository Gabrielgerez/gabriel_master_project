\makeatletter
\def\input@path{{../}}
\makeatother
\documentclass[../master_thesis.tex]{subfiles}
\begin{document}
\chapter{Computational Chemistry}
\section{\ac{SE}}

Consider a one electron system (such as \ce{H} or \ce{He+}). The system can be
described by a wavefunction $\Psi$. This wavefunction, by definition, has all
information about the system, which can be extracted by solving the Eigenvalue
equation with a characteristic operator $\vartheta$ for an observable property
$ e $ to be extracted \cite{Cramer:2004}:
\begin{equation}
  \vartheta\Psi = e\Psi\label{eq:eigenequation}
\end{equation}

The information that is of most interest in computational chemistry is the total
energy of the system $ E $. The operator for the energy of the system is called
the Hamiltonian $H$ which contains terms for the kinetic $T$ and potential
energy $V$ of the system. Plugging this into \ref{eq:eigenequation} yields the
\ac{SE}:
\begin{equation}
  H\Psi = E\Psi\label{eq:SE}
\end{equation}



\begin{acronym}
\acro{AUS}[\href{https://www.sigma2.no/content/advanced-user-support}{AUS}]{Numerical Methods in Quantum Chemistry}
\acro{BO}{Born-Oppenheimer}
\acro{CTCC}[\href{http://www.ctcc.no}{CTCC}]{Centre for Theoretical and Computational Chemistry}
\acro{DC}{Dielectric Continuum}
\acro{DFT}{Density Functional Theory}
\acro{EFP}{Effective Fragment Potential}
\acro{EU}{European Union}
\acro{HF}{Hartree-Fock}
\acro{Hylleraas}[\href{https://www.mn.uio.no/hylleraas/english/}{Hylleraas}]{Hylleraas
  Centre for Quantum Molecular Sciences}
\acro{HPC}{High Performance Computing}
\acro{KTH}{Royal Institute of Technology}
\acro{LDA}{Local Density Approximation}
\acro{MCD}{Magnetic Circular Dichroism}
\acro{MCSCF}{Multiconfiguration Self Consistent Field}
\acro{MM}{Molecular Mechanics}
\acro{MW}{Multiwavelet}
\acro{NFR}{Norwegian Research Council}
\acro{NMQC}[\href{http://www.ctcc.no/events/conferences/2015/numeric-conference/}{NMQC}]{Numerical Methods in Quantum Chemistry}
\acro{NOTUR}[\href{https://www.notur.no/}{NOTUR}]{Norwegian Metacenter for Computational Science}
\acro{PCM}{Polarizable Continuum Model}
\acro{PI}{Primcipal Investigator}
\acro{QC}{Quantum Chemistry}
\acro{QM}{Quantum Mechanics}
\acro{QM/MM}{Quantum Mechanics/Molecular Mechanics}
\acro{ROA}{Raman Optical Activity}
\acro{SC}{semiconductor}
\acro{SCF}{Self Consistent Field}
\acro{SHG}{Second Harmonic Genertation}
\acro{STSM}{Short-term scientific mission}
\acro{TPA}{Two-Photon Absorption}
\acro{WP}{Work Package}
\acro{CBS}{Complete Basis Set}
\acro{TCG}{Theoretical Chemistry Group}
\acro{vdW}{van der Waals}
\acro{SE}{Schrödinger Equation}
\acro{PES}{Potential Energy Surface}
\acro{LCAO}{Linear Combination of Atomic Orbitals}
\acro{MRA}{Multi-Resolution Analysis}
\acro{NS}{Nonstandard}
\end{acronym}

\biblio
\end{document}

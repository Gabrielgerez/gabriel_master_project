\makeatletter
\def\input@path{{../}}
\makeatother
\documentclass[../Thesis.tex]{subfiles}
\begin{document}
\chapter{Solvent Effect}\label{chap:Solvent_effect}
In chapter \ref{chap:Quantum_chemistry} we saw how to approximate and
solve systems of molecules in vacuum. This was done using both \ac{HF} and \ac{DFT}
in iterative \ac{SCF} procedures. Results obtained from these methods would then
be used to determine certain properties, or observables, from these systems. Results
obtained from systems in vacuum can be called gas phase values.

The goal of these computations are to both to approximate systems and to manage to
make predictions in regards to these approximation. It makes sense then that one
of  the type of systems one would be most interested in approximating as a chemist
are solvation systems. This is because most chemical processes of interest in
chemistry, such as reaction mechanisms or acid-base reactions, happen in
solvent.

Our aim is to be able to simulate a solvent system in an appropriate way, such that
the models let us predict rates, mechanisms and other specific processes which
occur in solutions \cite{Tomasi:1994wt}.

A straightforward approach, consistent with the \ac{QM} methods defined up to now,
is to make a system a substrate atom surrounded by solvent molecules, all defined
by quantum mechanical methods. One can imagine the substrate to be a single atom
ion and the solvent to be water.  The first problem that arises is the quantity of
water molecules needed to simulate the solution system. A straight forward answer would
be to have enough water molecules to surround  the ion in one layer of them,
and just do our calculations from that point.

Let us put the molecules in a cube-like fashion around the cation, with each
water molecule in a face of the cube, totaling 6 water molecules. Our calculations
could be performed from this point, but one should see that the system is ordered,
and that it is not necessarily in a minimum with respect to the geometry. The
next course of action should be to perform a geometry optimization on these 7
molecules, or 19 nuclei. The first problem with this method appears in these
geometry optimization. Since we have so many nuclei, the \ac{PES} will be affected
by all of them simultaneously. This causes the \ac{PES} to have many small, local
minima, making it harder to find the global minimum of the surface \cite{Cramer:2004}. This is still
doable with full \ac{QM} methods, given that these type of calculations are done
on greater systems than just 19 nuclei, such as hydrocarbons, or small proteins.

Another question we can ask ourselves is how representative is this model to the
real system we are trying to simulate. After all, the water molecules on a solvent interact
with themselves and the substrate. As our model is now the water molecules are
only interacting with the ones beside them, while elsewhere there are no other water
molecules. Our model is representing a group of a substrate molecule surrounded
by water molecules on a vacuum. This is not what we are trying to model. We can add
more layers of water molecules, so as to simulate the interactions of water with
itself, but we will always have a set of water molecules facing vacuum.
We can add water molecules until the amount of molecules facing vacuum is negligible
with respect to the ones in a solvent environment. This will be an extremely big
system suffering of a \ac{PES} with many local minima and nuclei, making both
geometry optimizations and energy calculations expensive \cite{Jensen:2017}.

The system explained above is both expensive and not representative of the system we are trying
to model. Solvent models, presented in this chapter, present a solution to the dilemma above.
These solvents models, which will be the focus of this chapter, reference the case of a very dilute
solution, where we can assume the substrate is surrounded by the solvent only.

\section{Outlining the problem}
Solvation models describe one or more molecules (substrate) surrounded by
a set of other molecules (solvent). Most reactions of interest occur in a
solvent where the geometry, energy and kinetics of the reactants and products
are affected by their environment\cite{Mennucci:2018}.

We can describe the total Hamiltonian $\hat{H}_{sol}$ of a solvent-solute system as
\cite{Tomasi:1994wt}
\begin{equation}\label{eq:Hsolvent}
  \hat{H}_{sol} = \hat{H}_0 + \hat{V}_{sol},
\end{equation}
where $H_0$ is the gas phase Hamiltonian (in vacuo) and $\hat{V}_{sol}$ is the
potential arising from the interactions between solvent and solute. These
interactions can be considered as perturbations to the in vacuo system.
The \ac{SE} of the system is as follows \cite{Tomasi:1994wt}
\begin{align}\label{eq:solSE}
  \hat{H}_{sol}\Psi^{(f)} = E^{(f)}\Psi^{(f)},
\end{align}
where $f$ denotes the degree of  rigidity of the interactions (0 being with
completely static solvent molecules and rigidity decreasing with increasing
index). In most practical uses, we are interested in the polarization interaction
between the solvent and solute, meaning that one can replace the wave function $\Psi$ with the
total charge distribution of the solute $\rho_{\mathrm{tot}}$
\cite{Tomasi:1994wt}
\begin{align}
  \begin{split}\mathbb{}
      \rho_{\mathrm{tot}}(\rvec ; \vec{Q}) &=
      \rho_{\mathrm{nuc}}(\rvec, \vec{Q})
      + \rho_{\mathrm{el}}(\rvec ; \vec{Q}), \\
      \rho_{\mathrm{nuc}}(\rvec ; \vec{Q}) &=
      \sum_{\alpha} Z_{\alpha} \delta\left(\rvec
      - \vec{Q}_{\alpha}\right), \\
      \rho_{\mathrm{el}}\left(\vec{q}_{1} ; \vec{Q}\right) &=
      -\int\left|\Psi^{(f)}(\boldsymbol{q}, \vec{Q})\right|^{2}
      \mathrm{d}\vec{q}_{2} \ldots \mathrm{d}\boldsymbol{q}_{N_{\mathrm{el}}},
  \end{split}
\end{align}
where $\alpha$ is an index that iterates over all the nuclei of the molecule,
$Z_{\alpha}$ is the nuclear charge of each nucleus, $\vec{q}_1, ..., \vec{q}_{N_{el}}$
represents the coordinates of the electrons, $\vec{Q}_{\alpha}$ represents the
coordinates of the nuclei and $\rvec$ is a vector variable describing a point
in three dimensions.

There are four main interactions affected by solvent which one might be
interested in solving. These are electrostatic interactions, cavitation,
changes in dispersion and changes in bulk solvent structure \cite{Cramer:2004}.
In this thesis we work mostly with the electrostatics interactions of the
solute-solvent interface and the reaction field they create.

\subsection{Reaction field}\label{Reaction_field}
Let us consider a system where a solute $A$ with a dipole moment $\vec{\mu}_A$ is
introduced to a solvent $S$ where each solvent molecule $S_i$ has its own dipole
moment $\vec{\mu}_{S_i}$. On equilibrium the different $S_i$  would be placed
randomly, giving an average electric field of zero. When $A$ is introduced, its
$\vec{\mu}_A$ will induce an electric field that will affect all the
$\vec{\mu}_{S_i}$ of the $S_i$. The $S_i$ will reorient themselves so that their
$\vec{\mu}_{S_i}$ lie along the electric field induced by $\vec{\mu}_A$. This is
so that they are in a more energetically favorable position in the electric
potential of the field induced by $\vec{\mu}_A$, but in doing this, they will
act against their own electric field and lose conformational freedom, which will
cost energy. This will, in turn, change the electric field of $S$ so that it no
longer behaves uniformly. The electric field of $S$ will then affect $A$ so that
$\vec{\mu_A}$ lies along the new electric field of $S$ for the same reason
stated above, using free energy in doing so \cite{Cramer:2004}.

The process outlined above will repeat itself until the energy gain from
reorienting $A$ and $S$ is outweighed by the required energy of such a
reorientation. The energy at such a point is equal to half the total interaction
energy between $S$ and $A$ \cite{Cramer:2004}. The new field obtained at the
equilibrium position of this interaction is called the reaction field $U_r$ for
the interaction of $A$ and $S$.
This interaction can be represented as a statistical average over all degrees of freedom of
the interaction. This is done by replacing the solvent molecules with a continuous electric
field that is affected by the introduction of the substrate. This is done so
that one does not need to represent the charge distribution of the solvent explicitly \cite{Cramer:2004},
which can be expensive with bigger systems.
Solvent models that represent the charge distribution explicitly are called
explicit solvation models. The models that do not represent it explicitly are
implicit solvation models.

\section{Explicit solvation models}

The explicit method is necessary when the solvent molecules' geometry and states
are important to the measurement of the interactions of the substrate
\cite{Cramer:2004}.
One way to simplify the problem is to partition the system into two parts, the
solvent, modeled using molecular mechanics (MM) and the solute, modeled using
\ac{QM}, this method is called QM/MM \cite{Mennucci:2018}.

The main point of this model is that the energy contribution from the solvent
can be described as a sum of bonding and non-bonding contributions
\cite{Cramer:2004} where the solvent bonds are described as springs
\cite{Mennucci:2018} and the atoms as weights. The bonding contributions can be
divided into bending, stretching and torsion of the bonds. Both the bending
and the stretching of the bonds can be described with the Harmonic Oscillator
(HO) and the torsion with a periodic function \cite{Mennucci:2018}. The
non-bonding interactions can be described with Coulomb's law, or other equivalent
expressions of potentials, with the atoms as point charges. More information on
this can be read on \cite{Cramer:2004, Jensen:2017}.

\section{Implicit solvation models}
Most implicit models describe the solvent as a linear isotropic continuum
characterized by the static dielectric constant $\epsinf$ characteristic
of the bulk of the solvent \cite{Tomasi:1994wt, FossoTande:2013ka}. In this continuum
only the electrostatic interactions contribution are significant \cite{Tomasi:1994wt}.
The energy needed to create a cavity in the continuum in which the substrate is
contained and the dispersion and repulsion forces between the solvent and the
substrate often cancel out, or at least the contribution becomes negligible
\cite{Cramer:2004}. This means that the interaction potential $\hat{V}_{sol}$
in Equation \ref{eq:Hsolvent} becomes an electrostatic potential $\hat{V}_{\sigma}$
\cite{Tomasi:1994wt}
\begin{equation}
  \hat{V}_{\mathrm{int}}=\hat{V}_{\sigma}\left(\mathbf{q}, \mathbf{Q},
  \rho_{tot}, \epsinf\right)=\sum_{\alpha} Z_{\alpha} U_r
  \left(\mathbf{Q}_{\alpha}\right)-\sum_{i} U_r\left(\mathbf{q}_{i}\right),
\end{equation}
where $U_r$ is the reaction field potential.
The contribution of the interaction between the solvent and the substrate to the total energy
$E^{(f)}$ is given by
\begin{equation}
  W_{tot} = \int_{\real^3}\Psi^{(f)\star}\hat{V}_{\sigma}\Psi^{(f)}\text{d}q...\text{d}q_{N_{el}}
  = \braket{\rho_{tot}(\rvec)|U_r(\rvec)}.
\end{equation}
This means that determining the reaction potential as well as knowing the coordinates
of the different particles will solve the Equation \ref{eq:solSE}.
Since we are looking at the effect the charge distribution of a solute has on
the interaction potential, we set up a Poisson equation. We also know that the change in
the potential must also be related to the dielectric constant that characterizes
the continuum $\epsilon$. We try to find the solution to the following
electrostatic potential equation, also called the \ac{GPE}
\begin{equation}\label{eq:gpe}
  \nabla\big( \epsilon(\rvec) \nabla U(\rvec)\big) = -4\pi\rho_{tot},
\end{equation}
where $U(\rvec)$ is the electrostatic potential of the interaction between
the solute and the solvent.

There are two main ways of solving the \ac{GPE} that are implemented in \acp{PCM}.
These are either direct solutions of Equation \ref{eq:gpe}, or introducing boundary conditions
where the Volumes inside and outside the cavity are separated by the a two dimensional
cavity surface. In the latter method the problem is solved at the surface and is represented
as an apparent surface charge $\sigma$ distributed along the surface.

\subsection{Cavity}\label{Cavitytitle}
All continuum models make use of a cavity in which the solute $A$ resides
\cite{Tomasi:1994wt, Cramer:2004, Tomasi:2005ipa}. The shape of the cavity
must be defined so it includes the whole molecule $A$ and its charge distribution.
Although, since the charge distribution of any molecule persists to infinity there
will always be an overlap with the charge distributions of the medium in real
systems. The charge that is not contained within the cavity is often called
escaped charge \cite{Tomasi:2005ipa}.

The size of the cavity is critical to the calculated results of the model, if
the cavity is too big the solvation effects are dampened, if it is too small
various errors may arise in the computation of the interactions to be studied
\cite{Tomasi:1994wt}. The cavity shape should closely
resemble the shape of the molecule itself \cite{Tomasi:2005ipa}. There are two
main ways of defining the shape and size of the cavity: (1) regular
shapes and (2) molecular shapes.

Regular shapes such as spheres, cylinders and ellipsoids have the advantage of
faster and simpler computations than molecular shapes. The fact that the cavity
should resemble the molecule and its charge distributions still stands, and it
is not realistic to assume that all molecules have regular shapes (apart from
maybe single atoms and linear molecules) \cite{Tomasi:2005ipa}.

Molecular shapes give a better representation of the solute in the solvent due
to the fact that they more closely follow the shape of the solute molecule.
There are three main molecular shapes I wish to mention:
interlocking spheres, surfaces traced by solvent probes and isodensity surfaces.

\subsubsection{Molecular surface shapes}

\paragraph{The interlocking spheres model}
The interlocking spheres model consists of creating spheres centered at the nuclei
of the molecule. Other centers can be in functional groups, so that they envelope a whole group,
or in specially designated spaces around the molecule so that the give an accurate
representation of the parts of the substrate molecule the solvent has access to
\cite{Tomasi:1994wt}. The latter is explained more in the next section.

The consensus for the radius of each cavity $R_i$ is that it has to be close to the \ac{vdW} radius
$R_{vdw}$ of the atom the sphere is centered in. One of the most used set of radii for
this are the the radii provided by Bondi \cite{doi:10.1021/j100785a001, Tomasi:2005ipa}
and are generally  scaled by a coefficient $f$ that is close to 1. The following relation holds
\cite{Tomasi:1994wt}
\begin{equation}
  R_i = fR_{vdw},
\end{equation}
where $f$ has been found to be, by statistical analysis, $f = 1.2 \pm 0.1$ in order to
yield the best results. Its value is $1.2$ for water while it can vary with
other solvents \cite{Tomasi:1994wt}. Although this only stands for neutral
solutes. Some results imply that the best radius for cations is the covalent
radius and the best one for anions is the ionic radius \cite{Tomasi:1994wt}.
\paragraph{Solvent probe tracing surfaces}\label{Spts}
This family of surfaces are based on the interlocking spheres model in that a
solvent probe (normally a spherical probe) is rolled around the Interlocking
sphere surface to trace a new surface. The inwards facing face of the probe traces a surface
which the solvent cannot access. This surface is called the \ac{SES}
\cite{Tomasi:2005ipa, Mennucci:2018}.  One can think of this as
smoothing so as to not get polarization in areas around the molecule which should
not be accessible to the solvent. Both SAS and \ac{SES} are shown in the following
Figure \ref{fig:SASSEShomemade}
\begin{figure}[ht]
  \includegraphics[width=\linewidth]{img/SASSES.png}
  \caption{SAS and SES surfaces}
  \label{fig:SASSEShomemade}
\end{figure}

\paragraph{Isodensity surfaces}
A third type of surface is one defined by a certain value of the charge density.
This is usually defined by setting a threshold value of the charge density for
which the surface will be drawn (usually $0.0004-0.001\ \text{a.u.}$)\cite{Tomasi:2005ipa}.


\subsection{Boundary conditions}
As stated before, in continuum models the solvent is represented by an isotropic
continuum defined by the dielectric constant $\epsinf$ characteristic to the bulk
solvent \cite{Tomasi:1994wt}. The interface between the solute and the continuum
is the cavity. One way to think of the value of the continuum is by defining a
function epsilon with the cavity as a boundary condition \cite{Tomasi:2005ipa, Tomasi:1994wt}
\begin{equation}
  \epsilon(\rvec) =
  \begin{cases}
  \epsinf & \text{for} \ \rvec\in C\\
  1 & \text{for}\ \rvec \notin C
\end{cases},
\end{equation}
where $C$ is the cavity. We then solve the polarization problem at the surface
of the cavity. We define an Apparent surface charge $\sigma$ which defines the
polarization of the electric field as charges at the surface of the cavity. Apparent surface
charge methods are used to solve for the polarization.
 We define this problem by partitioning the volume of the system
into two parts, one inside the cavity and one outside the cavity \cite{Tomasi:1994wt}.
Another assumption that is made is to say that we have an ideal charge distribution
that does not reach outside the cavity \cite{Cramer:2004, Tomasi:1994wt}.
This means that \cite{Tomasi:1994wt}
\begin{equation}
  \rho_{\mathrm{tot}}(\rvec)=0 \quad \rvec \notin C.
\end{equation}
This lets us define the Poisson equation inside and outside the cavity as follows
\cite{Sorland, Tomasi:1994wt}:
\begin{align}
  \begin{split}
    \nabla^2U(\rvec) &= -4\pi\rho_{tot}(\rvec)\ \rvec\in C,\\
    \epsinf\nabla^2U(\rvec) &= 0 \ \rvec \notin C,
  \end{split}
\end{align}
where, for a point $\vec{s}$ and a normal outwards vector $\vec{n}$ on the
surface of the cavity, the following conditions are true
\cite{Sorland, Tomasi:1994wt}
\begin{align}
  \begin{split}
    U_{in}(\vec{s}) = U_{out}(\vec{s}),\\
    \frac{\partial U_{in}}{\partial\vec{n}} = \epsinf\frac{\partial U_{out}}{\partial\vec{n}},
  \end{split}
\end{align}
where $U(\rvec)$ is the sum of the potential of the substrate in vacuo
$U_{vac}$ and the reaction field potential $U_r$\cite{Sorland, FossoTande:2013ka}
\begin{equation}\label{eq:Utot}
  U(\rvec) = U_{vac}(\rvec) + U_r(\rvec).
\end{equation}

\subsection{Approaches to a solution}\label{approchessolv}
\subsubsection{Apparent surface charge}
In this family of solutions we look at the classical electrostatics description
of the problem. we do this by defining an apparent charge distribution
$\sigma$ confined to the surface of the cavity $\Gamma$ \cite{Tomasi:1994wt, Tomasi:2005ipa}.
The apparent surface charge distribution at the boundary between two regions $i, j$
is defined as
\begin{equation}
  \sigma_{ij} = -(\vec{P}_j - \vec{P}_i)\cdot\vec{n}_{ij},
\end{equation}
where $\vec{n}_{ij}$ is the unit vector pointing from $i$ to $j$ and $\vec{P}$ is
the polarization vector of a region of constant isotropic permitivity (such as
the continuum in \ac{PCM}) defined as
\begin{equation}
  \vec{P}_i (\rvec) = - \frac{\epsilon_i - 1}{4\pi}\nabla U(\rvec),
\end{equation}
where $\epsilon_i$ is the permitivity constant of region $i$ \cite{Tomasi:2005ipa}.
In a solvation model we only have two regions, one inside and one outside. We
know that $\epso = 1$ so $\vec{P}_0 = 0$ leaving only the outside region to
construct $\sigma$. This gives us \cite{Tomasi:2005ipa}
\begin{equation}\label{eq:asc}
  \sigma(\vec{s}) =\frac{\epsinf - 1}{4\pi\epsinf} \frac{\partial}{\partial\vec{n}} (U_{vac} + U_{\sigma})_{in}\ ;\ s \in \Gamma,
\end{equation}
and we find the reaction potential of the apparent surface charge $U_{\sigma}$
with the following integral
 \begin{equation}
   U_{\sigma}(\rvec) = \int_{\Gamma}\frac{\sigma(\vec{s})}{\abs{\rvec - \vec{s}}} \text{d}\vec{s}\ ;\ s \in \Gamma,
 \end{equation}
If we divide the surface into a set of finite elements, tesserae
\cite{Tomasi:2005ipa, Sorland}, with set areas $A$ that are small enough to be able
to assume that $\sigma$ does not change inside each of these tesserae, will let us define the integral
as a sum over all the tesserae
\begin{equation}
   U_{\sigma}(\rvec) \simeq \sum_k\frac{\sigma(\vec{s}_k) A_k}{\abs{\rvec - \vec{s}_k}} \text{d}\vec{s}
   = \sum_k\frac{q_k}{\abs{\rvec - \vec{s}_k}},
\end{equation}
where $q_k = \sigma(\vec{s}_k) A_k$ is the point charge of each tesserae
\cite{Tomasi:2005ipa}.
Many slightly different types of \ac{ASC} have been implemented as of now. Some of these
are \ac{COSMO}, \ac{IEF} and \ac{SVPE}.

\paragraph{\ac{COSMO}}
In this version of \ac{ASC} we proceed as normal except that we define $\epsinf = \infty$.
This way we determine the \ac{ASC} ($\sigma^{\star}$) by the local value of the electrostatic
potential instead of computing the normal component of its gradient \cite{Tomasi:2005ipa}.
We then scale $\sigma^{\star}$ as
\begin{equation}
  \sigma(\vec{s}) = \frac{\epsinf -1}{\epsinf + k}\sigma^{\star}(\vec{s}),
\end{equation}
where $k$ is small \cite{Tomasi:2005ipa}.

\paragraph{\ac{IEF}}
In this method we define the electrostatic potentials in terms of Green integral
functions. The Green function $G(\vec{x}, \vec{y})$ is the potential produced at
a point $\vec{x}$ by a unit charge at $\vec{y}$ \cite{Tomasi:2005ipa}. We define
the electrostatic potentials as follows

\begin{equation}
\begin{aligned} U(x) &=\int_{R^{3}} G^{s}(x, y) \rho_{\mathrm{M}}(y) \mathrm{d} y, \\
  U_{vac}(x) &=\int_{R^{3}} G(x, y) \rho_{\mathrm{M}}(y) \mathrm{d} y, \\
  U_{r}(x) &=\int_{R^{3}} G^{r}(x, y) \rho_{\mathrm{M}}(y) \mathrm{d} y, \end{aligned}
\end{equation}
where $G(\vec{x}, \vec{y})=1 /|x-y|$ is the Green kernel for the operator
$-\nabla^2$, $G^s(\vec{x}, \vec{y})$  is the Green function of
$-\nabla(\epsilon(\rvec)\nabla)$ and $G^{r}(x, y)=G^{\mathrm{s}}(\vec{x},
\vec{y})-G(\vec{x}, \vec{y})$. Then the reaction potential $U_r$ can be represented
as  \cite{Tomasi:2005ipa}

\begin{equation}
U_{\mathrm{r}}(x)=\int_{\Gamma} \frac{\sigma(\vec{y})}{|\vec{x}-\vec{y}|}
\mathrm{d} \vec{y} \forall \vec{x} \in R^{3},
\end{equation}
where the surface charge $\sigma$ is given by solving the following equation

\begin{equation}
\left[2 \pi\left(\frac{\epsilon+1}{\epsilon-1}\right)-D_{\mathrm{i}}\right] S_{\mathrm{i}} \sigma=-\left(2 \pi-D_{\mathrm{i}}\right) V_{vac},
\end{equation}
where $S_{\mathrm{i}}$ and  $D_{\mathrm{i}}$ are two of the four components of the Calderon projector \cite{Tomasi:2005ipa}.

\paragraph{\ac{SVPE}}
The main focus of this method is that the solute charge distribution has a trailing
"tail" that reaches outside the surface \cite{Tomasi:2005ipa}. We define the
reaction potential as consisting of a \ac{ASC} term $U_{\sigma}$ and a exterior
term $U_{\beta}$
\begin{align}
  \begin{split}
    U_r(\vec{x}) = U_{\sigma}(\vec{x}) + U_{\beta}(\vec{x}),\\
    U_{\beta}(\vec{x}) = -\frac{\epsinf - 1}{\epsinf}\int_{ext} \frac{\rho_{tot}(\vec{y})}{\abs{\vec{x} - \vec{y}}} \text{d}\vec{y},
  \end{split}
\end{align}
where $x, y$ are defined as above.

\subsubsection{Multipole expansion}
In this solution we write the total electrostatic potential $U(\rvec)$ in terms
of Legendre polynomials. This way the potential is reduced to a set of two
spherical harmonics functions, one outside and one inside the cavity.
\begin{align}
  \begin{split}
    U_{in}(\rvec) &= \sum_{l=0}^{\infty}\frac{1}{r^{l+1}}\sum^l_{m=-l}B_{lm}Y^m_l(\theta, \phi) \\
    &+ \sum_{l=0}^{\infty}\frac{(l+1)(\epsinf - 1)}{l + (l+1)\epsinf} \frac{r^{l+1}}{a^{2l+1}}\sum^l_{m=-l}B_{lm}Y^m_l(\theta, \phi),\\
    U_{out}(\rvec)
    &=\sum_{l=0}^{\infty}\frac{2l+1}{l+(l+1)\epsinf}\frac{1}{r^{l+1}}\sum^l_{m=-l}B_{lm}Y^m_l(\theta, \phi),
  \end{split}
\end{align}
where $Y^m_l$ is the spherical harmonics with angular momentum $l$ and projection
along the z-axis $m$, $a$ is the radius of the cavity and $B_{lm}$ is a set of
constants which need to be found \cite{Tomasi:1994wt}. It goes without saying that
since we are using spherical harmonics, the cavities used in this method must be
spherical. Some methods exist to work with ellipsoidal cavities and even interlocking
spheres, which are outlined in \cite{Tomasi:1994wt} and \cite{Tomasi:2005ipa}.


\subsection{Solving the generalized Poisson equation in multiwavelet basis}\label{solmw}
Following Fosso--Tande \cite{FossoTande:2013ka} the cavity is defined as a molecular shape
interlocking spheres cavity. We start by defining a normal
signed distance function $s_i$ of a point $\rvec$ away from the surface of the
sphere $C_i$ with radius $R_i = 1.2\cdot R^{vdw}_i$ and centered on atom $i$ with
coordinate $\rvec_i$ \cite{FossoTande:2013ka}
\begin{equation}
  s_i(\rvec) = \abs{r -r_i} - R_i.
\end{equation}
The cavity for the $i$-th atom is defined as
\begin{equation}
  C_i(\rvec) = 1 - \Theta(s_i(\rvec)) =\Theta(-s_i(\rvec)),
\end{equation}
where $\Theta$ is a heavyside-like function that enables to smoothly transition
from the inside of the cavity (value $1$), to the surface of the cavity (value
$\frac{1}{2}$) and to the outside of the cavity (value $0$)
\cite{Sorland, FossoTande:2013ka}. We use the error function $\erf(x)$ to define
$\Theta$ and a parameter $\sigma$ characterizing the width of the transition
between the inside and outside of the cavity
\begin{equation}
  \Theta\left(s_i\right) = \frac{1}{2}\left(1 + \erf\left(\frac{s_i}
  {\sigma}\right)\right).
\end{equation}
The function for the complete set of interlocking spheres $C$ for a molecule
with $N$ atoms is given by
\begin{equation}
  C(\rvec) = 1 - \prod_{i=1}^N(1 - C_i(\rvec)).
\end{equation}
This function can then be used further in creating a dielectric function
$\epsilon$ that switches smoothly from $\epso$ (permitivity of free space
with value $1$ \cite{FossoTande:2013ka}) to $\epsinf$ either with a linear
representation
\begin{equation}\label{eq:dielfuncexp}
  \epsilon(\rvec) = \epsinf  + (\epso - \epsinf)C(\rvec),
\end{equation}
or with an exponential representation
\begin{equation}\label{eq:dielfunclin}
  \epsilon(\rvec) =\epso\exp\left( \left(\log\frac{\epsinf}{\epso} \right)
  (1 - C(\rvec)) \right),
\end{equation}
which is better suited for the computation of its $\log$-derivative. This is important
because differentiating the dielectric function directly will give us a slope that is
not entirely centered at the cavity surface, which gives rise to instabilities in
the results \cite{FossoTande:2013ka}.  The log-derivative for the exponential representation
is defined as \cite{FossoTande:2013ka}
\begin{equation}\label{eq:logderepsexp}
  \nabla\log\epsilon(\rvec) = \frac{\nabla\epsilon(\rvec)}{\epsilon(\rvec)}
   = \left(\log\frac{\epso}{\epsinf} \right)\nabla C(\rvec).
\end{equation}
The log derivative of the linear dielectric function is expressed as
\begin{equation}\label{eq:logderepslin}
  \nabla \log\big(\epsilon (\rvec)\big) = \frac{\nabla\epsilon(\rvec)}{\epsilon(\rvec)}
  = \frac{(\epso - \epsinf)}{\epsilon(\rvec)}\nabla C(\rvec).
\end{equation}
With either of those dielectric functions, the \ac{GPE} can be constructed
\cite{Sorland, FossoTande:2013ka}
\begin{align}\label{eq:gpemw}
  \nabla\cdot \big(\epsilon(\rvec)\nabla U(\rvec)\big) &= -4\pi \rho_{tot}(\rvec), \\
  \nabla^2U(\rvec) &= -4\pi \frac{\rho_{tot}}{\epsilon(\rvec)} - \frac{\nabla
  \epsilon(\rvec)\cdot \nabla U(\rvec)}{\epsilon(\rvec)}.
\end{align}
We define the effective charge distribution $\rho_{eff}$ and the surface charge
distribution $\gamma_s$ \cite{FossoTande:2013ka}
\begin{align}\label{eq:effrhogamma}
  \rho_{eff}(\rvec) &= \frac{1}{\epsilon(\rvec)}\rho_{tot}(\rvec), \\
  \gamma_s(\rvec) &= \frac{1}{4\pi}\frac{\nabla\epsilon(\rvec)}{\epsilon(\rvec)}\cdot\nabla U(\rvec),
\end{align}
and substitute them into Equation \ref{eq:gpemw}
\begin{equation}\label{eq:gpepractical}
  \nabla^2U(\rvec) = -4\pi\big( \rho_{eff}(\rvec) + \gamma_s(\rvec)\big).
\end{equation}
Which, if we were to define $\rho$ as entirely contained in the cavity, would mean
that the polarization is completely defined by $\gamma_s$ and Equation \ref{eq:gpepractical}
would take the form of Equation \ref{eq:asc}. This means that $\gamma_s$ is equivalent to
$\sigma$.
We get the total iteration potential $U(\rvec)$ between the substrate and the solvent by
applying the Poisson operator $\hat{\mathscr{P}}$ as defined in Equation
\ref{eq:Poissonopmw} to equation \ref{eq:gpepractical}
\begin{equation}
  U(\rvec) = \hat{\mathscr{P}}\big( \rho_{eff}(\rvec) + \gamma_s(\rvec)\big).
\end{equation}
We can then find the reaction potential following Equation \ref{eq:Utot} and
rearranging it so we get the following
\begin{equation}
  U_r(\rvec) = U(\rvec) - U_{vac}(\rvec).
\end{equation}
The reaction field energy is given as \cite{FossoTande:2013ka}
\begin{align}
  E_{tot} &= - E_{vac} + \frac{1}{2}E_r, \\
  E_r &= \braket{U_r | \rho_{tot}}.
\end{align}
and $E_{vac}$ is the gas phase energy of the substrate. This means that we
can easily add the energy contributions from the reaction field to the \ac{SCF} energy
as a simple addition.

Since the solution of
the total potential has itself in both sides of the
equation we will need to iterate with an initial guess. This is called \ac{SCRF}
and will be implemented more thoroughly in Chapter \ref{chap:implementation}.

\section{Variational formulation of the Generalized Poisson equation}
Following Lipparini \cite{Lipparini:2010bg, Lipparini:2013} I'll arrive at a functional where
minimizing with respect to $U$ of the solvent energy contribution to
the total energy $G_s$ will give a way to determine $U_r$ variationally.

Starting from the \ac{GPE} defined from \cite{FossoTande:2013ka} we project it
into a suitable space with test functions $\psi$

\begin{align}
  \begin{split}\label{eq:projectedgpe}
    \nabla (\epsilon\nabla U) &= -4\pi\rho,\\
    -\braket{\psi|\nabla |\epsilon\nabla U} &= 4\pi \braket{\psi|\rho}.
  \end{split}
\end{align}
Applying the following vector identity
\begin{equation}
\nabla (\psi \epsilon\nabla U)=\nabla \psi \epsilon \nabla U +  \psi \nabla \epsilon \nabla U +  \psi \epsilon\nabla^{2} U,
\end{equation}
we get the integral
\begin{equation}
  -\int_{\real^3}\psi \nabla \epsilon \nabla U\text{d}r - \int_{\real^3} \psi\epsilon \nabla^{2} U\text{d}r = \int_{\real^3} \nabla \psi \epsilon \nabla U\text{d}r - \int_{\real^3} \nabla (\psi \epsilon\nabla U\text{d}r).
\end{equation}
The test functions vanish at infinity. Applying Gauss' theorem \cite{Lipparini:2013} we can turn the rightmost integral into
an integral on the surface, thus vanishing. This gives us
\begin{equation}
  -\braket{\psi|\nabla|\epsilon\nabla U} = \braket{\nabla \psi |\epsilon | \nabla U},
\end{equation}
which we substitute into Equation \ref{eq:projectedgpe}
\begin{equation}
  \braket{\nabla \psi |\epsilon | \nabla U} =  4\pi \braket{\psi|\rho},
\end{equation}
which lets us write the functional for the energy as \cite{Lipparini:2010bg}
\begin{equation}
  G(U) = \frac{1}{8\pi}  \braket{\nabla U |\epsilon | \nabla U} - \braket{U|\rho}.
\end{equation}
We now find the gradient of the functional in order to minimize with respect to $U$
\begin{align}
  \begin{split}
    \frac{\delta}{\delta U}G(U) &=  \diff{U}G - \nabla \cdot \diff{\nabla U}G \\
    &= -\nabla \frac{1}{8\pi}\diff{U}\braket{\nabla U|\epsilon|\nabla U} - \frac{\partial}{\partial U}\braket{U|\rho} \\
    \frac{\delta}{\delta U}G(U) &= \frac{1}{4\pi}\epsilon\nabla U - \rho
  \end{split}
\end{align}
where, if we set $\frac{\delta}{\delta U}G(U) = 0$ we get the \ac{GPE} from \cite{FossoTande:2013ka}
and thus we can use any type of minimization method to reach the right potential.

\input{acronyms.tex}
\biblio
\end{document}

\makeatletter
\def\input@path{{../}}
\makeatother
\documentclass[../master_thesis.tex]{subfiles}
\begin{document}
\chapter{Solvent Effect}
The goals of computational chemistry is to both approximate systems and to be
able to make predictions with regards to these approximations. In the previous
chapter I showed one of the ways to get good approximations of systems, now i
will talk about the set of systems that will be the focus for this thesis,
solvation models.
\section{Outlining the problem}
Solvation models describe one or more molecules (substrate) surrounded by
a set of other molecules (solvent). Most reactions of interest occur in a
solvent where the geometry, energy and kinetics of the reactants and products
are affected by their enviroment\cite{Mennucci:2018}.

Our aim is to be able to simulate this system in an appropiate way, such that
the models let us predict rates, mechanisms and other specific processes which
occur in solutions \cite{Tomasi:1994wt}.
One can describe the solvation model in either an explicit or implicit way. In
this thesis I will focus on the implicit models, although i will give some
background in the explicit model.
A straightforward approach to the model is to try to, for example, make a
system of a \ce{Li^+} cation surrounded by water molecules. Here arises two
questions; how many water molecules should we add, and, in which positions
should we place them. Certainly one could say that we could just have enough
water molecules to surround the cation in one layer of them,  and just do a
geometry optimization \cite{Jensen:2017} on the system in order to get the
orientation and position of these water molecules.

Lets us put the molecules in a cube-like fashion around the cation, with each
water molecule in a face of the cube, totaling 6 water molecules. In total we
would have 7 molecules, 19 nucleus and 62 electrons to optimize over. Using the
\ac{BO} approximation and \ac{DFT} we could solve the electronic part of the
problem in only three dimensions, but we would still need to work with
$19\cdot3d$ dimensions for the geometry of the nuclei. Even if we did do this
optimization and calculated the different properties of this system, we would
find out that this approximation is very far from the truth.

If we look at the system we can see that this set of 7 molecules is suspended in
vaccuum, which is not what we are trying to simulate. the water molecules at
the edge would not behave as if they were in solution, but as if they were in
gas phase. We are trying to simulate a \ce{Li^+} cation in a water solution.
We can add additional layers of water molecules to diminish the error. But
just adding a single layer of water molecules in a cubelike manner would
increase the amount of molecules to $31$ with a nuclear wavefunction of
$31\cdot3d$ dimensions. Adding more layers one after the other would end up
reducing the error to a neglectable level, as the amount of water molecules on
the surface will scale quadratic manner compared to the cubic scaling of the
water molecules inside the volume. This also means that the size of the system
would be too big to realistically work with in \ac{QM} and we would be forced to
calculate the effects as statistical averages over phase space
\cite{Cramer:2004}.

\section{Explicit solvation models}

The explicit method is necessary when the solvent molecules' geometry and states
are importnant to the measurement of the properties of the substrate
\cite{Cramer:2004}.
One way to simplify the problem is to partition the system into two parts, the
solvent, modeled using molecular mechanics (MM) and the solute, modeled using
\ac{QM}, this method is called MM/QM \cite{Mennucci:2018}.

The main point of this model is that the energy contribution from the solvent
can be described as a sum of bonding and nonbonding contributions
\cite{Cramer:2004} where the solvent bonds are described as springs
\cite{Mennucci:2018} and the atoms as weights. The bonding contributions can be
dividided into bending, stretching and torsion of the bonds. Both the bending
and the strecthing of the bonds can be described with the Harmonic Oscillator
(HO) and the torsion with a periodic function \cite{Mennucci:2018}. The
non-bonding interactions can be described with Coulombs law with the atoms as
point charges. More information on this can be read on \cite{Cramer:2004,
Jensen:2017}.


\section{Implicit solvation models}
In this family of solvation models we describe the solvent effect as a
perturbation to the in vacuo energy of the solute. The way we do this is by
describing the solute as being surrounded by a polarizable continuum
characteristic to the solvent. This model is called the \ac{PCM}. We can
describe the total Hamiltonian $H_{tot}$ of such a system as
\cite{Tomasi:1994wt}:
\begin{equation}
  \hat{H}_{tot} = H_0 + V_{sol}
\end{equation}
where $H_0$ is the gas phase Hamiltonian (in vacuo) and $H_{solvation}$ is the
Hamiltonian describing the perturbation that is the solvent effect. There are
four main properties affected by solvation effect. These are Electrostatic
interactions, cavitation, changes in dispersion and changes in bulk solvent
structure \cite{Cramer:2004}.
In this thesis we work mostly with the electrostatics interactions of the
solute-solvent interface. They describe the reaction field.

\subsection{Reaction Field}
If a solute $A$ with a dipole moment is introduced to a solvent that has its
own dipole moment the molecules of the solvent will reorient themselves so that
they oppose the dipole moment of the $A$. When this happens the average electric
field of the solute is changed. This will itself act on $A$ and increase its dipole
moment and reorient itself in regards to the electric field of the solvent in a
more energetically favorable way. The solvent will repeat the same process with
the electric field induced by $A$. The reorienting, since it is a movement
that act against the electric field of the solvent, will cost energy. This
process will repeat until the energetic advantage of reorienting is outweighed
by the energy needed to reorient itself \cite{Cramer:2004}.



\subsection{Interlocking spheres}
\subsubsection{Generalized poisson equation}
\section{Solving the Problem}
\subsection{\ac{SCF} for The Generalized \\ Poisson equation}
\begin{acronym}
\acro{AUS}[\href{https://www.sigma2.no/content/advanced-user-support}{AUS}]{Numerical Methods in Quantum Chemistry}
\acro{BO}{Born-Oppenheimer}
\acro{CTCC}[\href{http://www.ctcc.no}{CTCC}]{Centre for Theoretical and Computational Chemistry}
\acro{DC}{Dielectric Continuum}
\acro{DFT}{Density Functional Theory}
\acro{EFP}{Effective Fragment Potential}
\acro{EU}{European Union}
\acro{HF}{Hartree-Fock}
\acro{Hylleraas}[\href{https://www.mn.uio.no/hylleraas/english/}{Hylleraas}]{Hylleraas
  Centre for Quantum Molecular Sciences}
\acro{HPC}{High Performance Computing}
\acro{KTH}{Royal Institute of Technology}
\acro{LDA}{Local Density Approximation}
\acro{MCD}{Magnetic Circular Dichroism}
\acro{MCSCF}{Multiconfiguration Self Consistent Field}
\acro{MM}{Molecular Mechanics}
\acro{MW}{Multiwavelet}
\acro{NFR}{Norwegian Research Council}
\acro{NMQC}[\href{http://www.ctcc.no/events/conferences/2015/numeric-conference/}{NMQC}]{Numerical Methods in Quantum Chemistry}
\acro{NOTUR}[\href{https://www.notur.no/}{NOTUR}]{Norwegian Metacenter for Computational Science}
\acro{PCM}{Polarizable Continuum Model}
\acro{PI}{Primcipal Investigator}
\acro{QC}{Quantum Chemistry}
\acro{QM}{Quantum Mechanics}
\acro{QM/MM}{Quantum Mechanics/Molecular Mechanics}
\acro{ROA}{Raman Optical Activity}
\acro{SC}{semiconductor}
\acro{SCF}{Self Consistent Field}
\acro{SHG}{Second Harmonic Genertation}
\acro{STSM}{Short-term scientific mission}
\acro{TPA}{Two-Photon Absorption}
\acro{WP}{Work Package}
\acro{CBS}{Complete Basis Set}
\acro{TCG}{Theoretical Chemistry Group}
\acro{vdW}{van der Waals}
\acro{SE}{Schrödinger Equation}
\acro{PES}{Potential Energy Surface}
\acro{LCAO}{Linear Combination of Atomic Orbitals}
\acro{MRA}{Multi-Resolution Analysis}
\acro{NS}{Nonstandard}
\end{acronym}

\biblio
\end{document}

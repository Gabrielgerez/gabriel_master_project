\makeatletter
\def\input@path{{../}}
\makeatother
\documentclass[../master_thesis.tex]{subfiles}
\begin{document}
\chapter{Quantum Chemistry}\label{chap:Quantum_chemistry}
\section{Quantum Mechanics}

\subsection{The Postulates of \ac{QM}}

\ac{QM} are based on a set rules that define operations and states. We will
present these rules as six different postulates of \ac{QM} (sometimes divided
as 5) \cite{Atkins:2011, Cohen:1973}.

\subsubsection{First Postulate}
The first postulate states that everything we can know about a physical system
can be extracted from the wave function $\Psi(x, t)$ of that system
\cite{Atkins:2011}. Additionally at a time $t_0$ the system is defined by
a state vector (or wave function) $\ket{\Psi(t_0)}\in L^2$ that has a defined
finite scalar product as \cite{Cohen:1973}:
\begin{equation}
  \braket{\Psi|\Psi} = \|\ket{\Psi}\|^2 =  \int_{\mathbb{R}^3}\Psi^\star\Psi d\vec{r}
\end{equation}

\subsubsection{Second Postulate}
A generic observable $O$ is represented by the application of a generic operator
$\hat{O}$. Two such observables are the position and momentum of a particle. These are
represented by $\hat{q_i}$ and $\hat{p_i}$ respectively, where
$i = \{x, y, z\}$. These operators fulfill the following commutation relations
\cite{Atkins:2011, Cohen:1973}:
\begin{align}
  \begin{split}
    [q_i, p_j] &= i \hbar \delta_{ij}\\
    [q_i, q_j] &= 0 \\
    [p_i, p_j] &= 0
  \end{split}
\end{align}

The operators that represent observables are hermitian, meaning that
\cite{Cohen:1973}:
\begin{equation}
  \hat{O} = (\hat{O}^{\star})^T = (\hat{O}^T)^{\star} = \hat{O}^{\dagger}
\end{equation}

These operators are linear as well:
\begin{equation}
    \hat{O}(f + g) = \hat{O}f + \hat{O}g\label{eq:oplinearity}
\end{equation}
Where $f$ and $g$ are functions.

\subsubsection{Third Postulate}
When measuring an observable $O$ on a system $\ket{\Psi}$, The only possible
values of the observable are eigenvalues of the corresponding operator $\hat{O}$
unto the measured system $\Psi$ at its current state $\ket{\Psi_i}$. The \eivals
are solutions to the following equation \cite{Cohen:1973}
\begin{equation}
  \hat{O}\ket{\Psi_i} = o_i \ket{\Psi_i}
\end{equation}
This holds true even if the state measured is not an eigenstate of $\hat{O}$.

We can represent a system state vector $\ket{\Psi}$ that is not an eigenstate
of $\hat{O}$ as a linear combination of eigenstates $\ket{i}$ of the operator
\begin{equation}
  \ket{\Psi} = \sum_i c_i \ket{i}
\end{equation}
where the coefficients $c_i$ are computed as the projection of $\ket{\Psi}$ onto
the eigenstates of the operator.
\begin{equation}\label{eq:projcoeff}
  c_i = \braket{i|\Psi}
\end{equation}
Let us say that we apply the generic operator on one of its eigenstates that
is multiplied with the projection coefficient $c_i$ from Equation \ref{eq:projcoeff}.
This gives us
\begin{equation}
  \hat{O}c_i\ket{i} = c_i\hat{O}\ket{i} = c_i o_i\ket{i} = o_i c_i\ket{i}
\end{equation}
Since $c_i\ket{i}$ is one of the components in the eigenstate $\ket{\Psi}$
applying the operator on it will always give us exactly one of its \eivals, with
different probabilities  depending on its projection on the set of eigenstates
of the operator \cite{Cohen:1973}.

\subsubsection{Fourth Postulate}
The fourth postulate states the possible probabilities of getting a specific
\eival for a given measurement.
Let $\ket{\Psi_i}$ be an \eifunc of $\hat{O}$ such that:
\begin{equation}
  \hat{O}\ket{\Psi_i} = o_i\ket{i}
\end{equation}
and
\begin{equation}
  \ket{\Psi} = \sum_i c_i \ket{i} \label{eq:lincomb}
\end{equation}
has no degenerate \eival,
The probability of measuring \eival  $o_i$ from $\ket{\Psi}$ is given by
\cite{Cohen:1973}:
\begin{align} \label{eq:post4}
  \begin{split}
    \mathscr{P}(o_i) &= \abs{\braket{i|\Psi}}^2\\
                     &= \sum_{j}c_i^\star c_j\braket{i|j}\\
                     &= \sum_{j}c_i^\star c_j\delta_{ij}\\
                     &= \abs{c_i}^2
  \end{split}
\end{align}

Following the equation \ref{eq:post4} the probability of measuring the
\eival $o_i$ from $\ket{\Psi_i}$ is just one. The sum of all the probabilities
for each \eival is:
\begin{equation}
  \sum_i\mathscr{P}(o_i) = \sum_i \abs{c_i}^2 = 1
\end{equation}
Here we have ignored the case for degenerate \eival and continuous spectra of
\eival, where the method is analogous to the one used above. The reader is
invited to look them up themselves in \cite{Cohen:1973, Atkins:2011}.

\subsubsection{Fifth Postulate}
The fifth postulate states that immediately after a measurement where the \eival was $o_i$
the state of the system $\ket{\Psi}$ collapses into a state where the only value one
can measure is $o_i$, that is $ \ket{\Psi_i} $ using the conventions stated in postulate 3 and 4.
This is because before measuring, the probabilities for any \eival
is as stated in the fourth postulate. When the value has been measured, though,
the uncertainty does not exists, as the state of the system must be one that gives
exactly that value. The following Equation \ref{eq:post5} represents the postulate.
\begin{equation}\label{eq:post5}
  \ket{\Psi} \stackrel{o_i}{\Rightarrow} \ket{\Psi_i}
\end{equation}

\subsubsection{Sixth Postulate}
The system when undisturbed changes in a deterministic way \cite{Cohen:1973}.
This change is governed by the time dependent \SE \cite{Cohen:1973, Atkins:2011}:
\begin{equation}\label{eq:tdepSE}
  i\hbar\frac{\partial}{\partial t} \ket{\Psi} = \hat{H}\ket{\Psi}
\end{equation}
Where $\tilde{H}$ is the Hamiltonian operator which has the total energy of the
system as its \eivals.

\subsection{The \SE}
Lets consider the following \SE for a particle allowed to move in only one dimension
and where its potential energy varies with position (e.g. the Harmonic oscillator
model \cite{Cohen:1973, Atkins:2014}).
\begin{equation}
  \hat{H}\Psi = \left(-\frac{\hbar^2}{2m}\frac{\partial^2 }{\partial x^2} + \hat{V}(x)\right)\Psi = i\hbar\frac{\partial}{\partial t} \Psi\label{eq:1DSE}
\end{equation}
We can substitute $$\Psi(x, t)=\psi(x)\tau(t)$$ into \ref{eq:1DSE}
by assuming that the wave function can be separated into spatial and time dependent
functions \cite{Atkins:2011}:
\begin{align}
  \begin{split}\label{eq:sepvar1DSE}
    \left(-\frac{\hbar^2}{2m}\frac{\partial^2 }{\partial x^2} + \hat{V}(x)\right)\psi\tau &= i\hbar\frac{\partial}{\partial t} \psi\tau \\
    -\frac{\hbar^2}{2m}\tau\frac{d^2 \psi}{d x^2} + \hat{V}(x)\psi\tau &= i\hbar\psi\frac{d\tau}{d t}\\
    -\frac{\hbar^2}{2m}\frac{1}{\psi}\frac{d^2\psi }{d x^2} + \hat{V}(x) &= i\hbar\frac{1}{\tau}\frac{d\tau}{d t}
  \end{split}
\end{align}
In the last step of the Equation \ref{eq:sepvar1DSE} we divided both sides with $\frac{1}{\tau\psi}$.
This shows us that, since the left-hand side of the equation is only dependent on $x$ and the right-hand
side is only dependent on $t$, no matter how much we change the each of the coordinates, they must always equal to
a constant. This constant will be denoted by $E$ as it is the energy of the system. This gives us the following set of equations \cite{Atkins:2011}:
\begin{subequations}
  \label{eq:sysSE}
  \begin{align}
    -\frac{\hbar^2}{2m}\frac{d^2\psi}{d x^2} + \hat{V}(x)\psi &= E\psi  \label{eq:timeindepWF}\\
    i\hbar\frac{d\tau}{d t} &= E\tau  \label{eq:timedepWF}
  \end{align}
\end{subequations}
Equation \ref{eq:timedepWF} can be solved by observation as \cite{Atkins:2011, Cohen:1973} :
\begin{equation}
  \tau(t) = e^{-iE\frac{t}{\hbar}}
\end{equation}
while the remaining Equation \ref{eq:timeindepWF} can be rewritten as
\begin{equation}
  \hat{H}\psi = E\psi\label{eq:timeindepSE}
\end{equation}
Which is the time-independent \SE.

The energy of the system can be calculated as follows, assuming that the wave
function $\Psi$ \cite{Cramer:2004}:
\begin{equation}\label{eq:endevaleq}
  \frac{\braket{\psi|\hat{H}|\psi}}{\braket{\psi|\psi}} = E
\end{equation}
If the wave function is normalized the denominator becomes $1$.

\section{Two particle system}
A two particle system (such as the \ce{H} or the \ce{He^+} atoms) has a simple
hamiltonian of the form
\begin{equation}
  \hat{H}=\hat{T}_{N}+\hat{T}_{e}+\hat{V}
\end{equation}
where $\hat{T}_N$ and $\hat{T}_e$ are the kinetic energy operators of the nucleus $N$ and
of the electron $e$ and $\hat{V}$ is the Coulomb potential for two particles
\cite{Atkins:2014, Jensen:2017}.
This system has an analytical solution in which one follows the following set
of steps.
\begin{enumerate}
  \item Set a center of mass coordinate system.
  \item Change to spherical coordinates so that the potential operator becomes a
  simple function of the radius.
  \item Separate into radial function $R(r)$ and angular function
  $Y(\theta, \phi)$. The angular function can be separated into two more
  functions $\Theta(\theta)$ and $\Phi(\phi)$.
  \item Solve as three sets of differential equations \cite{Simons:2016}.
\end{enumerate}
Following the steps above, the 2 particle system is analytically solvable when
the potential energy is defined as function of only the distance between the
nucleus and the electron.

\section{Many body systems}

For bigger systems, there is no practical way to analytically solve the
\SE \cite{Jensen:2017}. For one, for each particle, the amount of dimensions that need to be
evaluated increases by three, that is, one can expect the wave function
dimension to increase by a factor of $3N$ for each particle $N$
\cite{Cramer:2004}.

Additionally, the potential energy operator becomes more complicated,as it
would not just have the attractive forces between electron-nucleui,but also the
repulsive forces between all the electrons and between all the nuclei. Both of
these problems add more terms per particle and thus would be impossible to be
solved in a realistic time frame \cite{Jensen:2017}.

\subsection{\ac{BO} approximation} \label{BO}
A many body system consists of $N$ nuclei with mass $m_I$ for each nucleus $I$
and $n$ electrons with mass $m_i$ for each electron $i$. Each nucleus has a
charge $Z_Ie$ and each electron has a charge $-e$, where $Z_I$ is the atomic
number of the nucleus $I$  and $e$ is the Elementary charge \cite{Atkins:2014}. The $N$ nuclei and $n$
electrons are located in a three dimensional coordinate system where each nucleus $I$ has
coordinates $\vec{R}_I = (x_I, y_I, z_I)$ and each electron has coordinates
$\rvec_i = (x_i, y_i, z_i)$.

We are trying to calculate the total energy and wave functions of the system. For
that we define a time independent \SE where the wave function is dependent on
the coordinates of both the electrons and the nuclei, which we will use a single $\vec{R}$
to the coordinates of all the nuclei and a single $\rvec$ to denote the coordinates
of all the electrons.
\begin{equation}\label{eq:totalheliumSE}
  \hat{H}\Psi(\rvec, \vec{R}) = E \Psi(\rvec, \vec{R})
\end{equation}

Here the Hamiltonian, as with the two particle system, has a potential energy $\hat{V}$ and a
kinetic energy $\hat{T}$ operator. As with the two particle system, we can divide $\hat{T}$ as
a sum of two contributions, an electron contribution $\hat{T}_e$ and a nuclear contribution
$\hat{T}_N$. The only difference is that these contributions are sums over all
the particles  instead of just one each\cite{Cramer:2004}
\begin{align}
  \hat{T}_e &= \sum_i^n \frac{\hbar}{2m_i}\nabla^2_i\\
  \hat{T}_N &= \sum_I^N \frac{\hbar}{2m_I}\nabla^2_I
\end{align}
where $\nabla^2_k$ is the Laplacian operator operating on particle $k$
\begin{equation}
  \nabla^2_k = \left( \ddiff{x_k} + \ddiff{y_k} + \ddiff{z_k} \right)
\end{equation}.

The potential operator, which we define as a sum of Coulomb potentials, is now much more
complicated. It consists of three contributions: the nucleus-electron attraction
$\hat{V}_{Ne}$, the nucleus-nucleus $\hat{V}_{NN}$ repulsion and the electron-electron
repulsion $\hat{V}_{ee}$ \cite{Cramer:2004}.
\begin{align}
  \hat{V}_{Ne} &= \sum^N_I\sum^n_{i}-\frac{Z_Ie^2}{\abs{\vec{R}_I - \rvec_i}}\\
  \hat{V}_{NN} &= \frac{1}{2}\sum^N_{I \neq J}\frac{Z_IZ_Je^2}{\abs{\vec{R}_I - \vec{R}_J}}\\
  \hat{V}_{ee} &= \frac{1}{2}\sum^n_{i \neq j} \frac{e^2}{\abs{\rvec_i - \rvec_j}}
\end{align}
where $I,\ J$ iterate through the nuclei and $i, \ j$ iterate through the electrons.
We can see now that the Hamiltonian is then dependent on the positions of all the electrons and all
the nuclei \cite{Jensen:2017}
\begin{equation}
  \hat{H}(\rvec, \vec{R}) = \hat{T}_e + \hat{T}_N + \hat{V}_{Ne} +\hat{V}_{NN} + \hat{V}_{ee}
\end{equation}
We can see that both repulsion potentials $\hat{V}_{NN}, \hat{V}_{ee}$ have $(N - 1)$ and $(n - 1)$ terms
 respectively, while the other terms have much simpler sums {sums over one index in the case of the
kinetic energies and $N\cdot n$ terms in the attraction potential $\hat{V}_Ne$}.

In the \ac{BO}  we assume that since the nuclei of the molecule are more massive
than the electrons (a single proton is $1836.152 673 89$ times more massive
than a proton \cite{NIST:2019}) that the electrons can instantaneously respond to
any change in the configuration of the nuclei \cite{Atkins:2011}. That means that we can solve an
electronic problem for any given nuclear geometry as if the nuclei were static
\cite{Cramer:2004, Jensen:2017, Atkins:2014}

Following this  assumption we separate the wave function into an electronic
$\Psi_e$ and a nuclear wave function $\Psi_N$
\begin{equation}
  \Psi(\rvec, \vec{R}) = \Psi_N(R)\Psi_e(\rvec; \vec{R})
\end{equation}.
Notice that the nuclear wave function is only dependent on the nuclear coordinates,
while the electronic has both the electron and nuclear coordinates as input. From
our assumption above we say that we solve an electronic \SE for each geometry
of the molecule, thus the nuclear coordinates  are parametric variables of
the electronic wave function (symbolized by the semicolon divider $;$) which remain
constant for each solution of the \SE.

The electronic \SE is as follows
\begin{equation}
    (\hat{H}_e(\rvec; \vec{R}) )\Psi_e(\rvec;\vec{R}) = E_e(\vec{R})\Psi_e(\rvec;\vec{R})
\end{equation}
Where the Electronic energy $E_e$ becomes a function of $\vec{R}$ which is solved as
a constant for each nuclear geometry. In the electronic Hamiltonian
we assume that the kinetic energy of the nuclei is zero, but we still need to
compute the nuclear repulsion, but as we are solving for any given geometry at a time
we can safely assume that it is a constant \cite{Cramer:2004}. The electron
Hamiltonian $\hat{H}_e$ is as follows
\begin{equation}\label{eq:elhamiltonian}
    \hat{H}_e(\rvec; \vec{R}) = \hat{T}_e + \hat{V}_{Ne} + \hat{V}_{ee}  + \hat{V}_{NN}
\end{equation}
The nuclear repulsion $V_{NN}$, since it is dependent on the nuclear geometry only,
can be treated as a constant.
Solving the electronic \SE for all possible nuclear geometries will give us
a \ac{PES} defined by the electronic energy for all the different geometries.
With this we can go on to solve the total \SE \cite{Jensen:2017}
\begin{equation}
  (T_N + E_e(\vec{R}))\Psi_N(\vec{R}) = E_{tot}\Psi_N(\vec{R})
\end{equation}

In the following sections we will be talking mostly about solving the electronic \SE,
 which means that we will drop the subscript and simply denote the electronic
wavefunction as $\Psi$ and its Hamiltonian as $hat{H}$.

\subsection{Variational Principle}
%need to edit this
Consider a complete set of orthonormal \eifuncs $ \Psi_i$  of the
Hamiltonian $H$. From Equation \ref{eq:lincomb} we build a representation of an
arbitrary wave function $\Phi$ with a linear combination of the \eifuncs with
projection coefficients $c_i$.
\begin{equation}
  \Phi = \sum\limits_ic_i\Psi_i
\end{equation}
For ease of notation we assume that  $\Phi$ is normalized. This means that the
following holds
\begin{equation}
  \braket{\Phi|\Phi} = \sum\limits_i^nc_i^2 = 1
\end{equation}
 Additionally, the energy associated with the Hamiltonian can
be calculated as in Equation \ref{eq:endevaleq} to give us this:
\begin{equation}
  \braket{\Phi|H|\Phi} = \sum\limits_i^nc_i^2E_i\label{eq:genenerg}
\end{equation}
This tells us the the energy of the wave function $\Phi$ can be determined by
knowing the energies $E_i$ of each \eifunc $\psi_i$ and the coefficients
$c_i$ associated with the linear combination that describes $\Phi$
\cite{Cramer:2004}.

We know that for this to be a quantum mechanical system there must be a lowest
energy among all energies $E_i$. We choose to call this lowest energy $E_0$.
Subtracting $E_0$ from Equation \ref{eq:genenerg} to find out the difference
between the calculated energy of the arbitrary wave function $\Phi$ with respect
to the ground state energy gives us:

\begin{equation}
   \sum\limits_i^nc_i^2(E_i - E_0) = \braket{\Phi | H | \Phi} - E_0
\end{equation}
We know that each term $c_i$ must be greater or equal to zero (non-trivial) and
that the term $(E_i - E_0)$ must be greater or equal to zero as well
\cite{Cramer:2004}, because each individual $E_i$ may add more energy to the
ground state. This leads to the following set of inequalities.

\begin{equation}
  \begin{split} \label{eq:varprin}
    \braket{\Phi | H | \Phi} - E_0 & \geqslant 0 \\
    \braket{\Phi | H | \Phi} & \geq E_0
  \end{split}
\end{equation}
We know from Equation \ref{eq:endevaleq} that the inequality in the last term in
\ref{eq:varprin} shows that the energy calculated as an \eival of $\Phi$ is
always greater or equal to zero. This lets us construct our trial wave functions
for the ground state of a system with any basis set. We can assess the quality
of the guess by their associated energies, attempting to reach as low a value
as possible \cite{Cramer:2004}.

\subsection{Self Consistent Field}
In the previous sections we showed two main points about solving the \SE for
many body systems. The first one, with the \ac{BO} approximation, is that we,
instead of solving for the energy of the whole system at the same time, can solve an electronic
\SE for each geometry of the molecule. The second point is that we can construct
the wave function with a basis set representation of our choice, for which its
accuracy is evaluated by how low the calculated energy is.

Our goal now is to find a way to systematically create wavefunctuions and
minimize them as in the variational principle. This is done with an iterative
procedure called the \ac{SCF}.

\subsubsection{Slater determinant}
The electron wave function seen in section \ref{BO} needs to include all the
coordinates that identify an electron. These properties are the spatial coordinates
$x, y,\  \text{and}\ z$, and the spin coordinate $s$. Another characteristic is that
it needs to be antisymetric, that is it must change sign whenever the coordinates
of two electrons, spatial and spin, are interchanged\cite{}.

The characteristics stated above can be summarized by writing the wave function as
a Slater determinant as \cite{Cramer:2004, Jensen:1999}
\begin{equation}
  \Psi_{SD} = \frac{1}{\sqrt{n!}}
  \begin{vmatrix}
    \phi_1(1) & \phi_2(1) & \ldots & \phi_n(1)\\
    \phi_1(2) & \phi_2(2) & \ldots & \phi_n(2)\\
    \vdots & \vdots & \ddots & \vdots \\
    \phi_1(n) & \phi_2(n) & \ldots & \phi_n(n)
  \end{vmatrix},
\end{equation}
where each $\phi_i$ is a one-electron spinorbital, a wave function constructed as
a product of a spatial function and a spin function\cite{Jensen:1999}, $n$ is the total amount of
electrons, the coefficient outside the determinant is a normalization constant
and each of the spin orbitals are orthonormal to each other.

In a Slater determinant each of the columns represent a spinorbital each and wthe rows the electron coordinates.
The spatial orbitals can, and are mostly, built using a basis set as decribed in
Equation \ref{eq:lincomb}. The spin functions are orthonormal \eifuncs of the operator $\hat{S}_z$
and have only two eigenvalues $\pm \frac{\hbar}{2}$ \cite{Cramer:2004}.

In certain situations it might be needed to express a wavefunction as a sum of Slater
determinants, but here we will work only as if our wave function can be constructed by
a single determinant wave function.

\subsubsection{Energy of a Slater determinant}

The energy of a single slater determinant is expressed as
\begin{equation}
  E = \sumdef[n]{i} \braket{\phi_i|\hat{h}|\phi_i} + \frac{1}{2} \sumdef[n]{ij}\left(\braket{\phi_j|\hat{J}_{i}|\phi_j}
  - \braket{\phi_j|\hat{K}_{i}|\phi_j}\right) + \hat{V}_{N}\text{,}\label{eq:SD}
\end{equation}
where
\begin{align}
\hat{h}\ket{\phi_i} &= \left(-\frac{1}{2}\nabla^2_i - \sumdef[N]{I}\frac{Z_I}{\abs{\Rvec_I - \rvec_i}}\right)\ket{\phi_i},\\
\hat{J}_{i}\ket{\phi_j(2)} &= \left\langle\phi_i(1)\left|\frac{1}{\abs{\rvec_1 - \rvec_2}}\right|\phi_i(1)\right\rangle \ket{\phi_j(2)},\\
\hat{K}_{i}\ket{\phi_j(2)} &= \left\langle\phi_i(1)\left|\frac{1}{\abs{\rvec_1 - \rvec_2}}\right|\phi_j(1)\right\rangle\ket{\phi_i(2)},
\end{align}
$\hat{V}_N$ is the nuclear repulsion, the operator $\hat{h}_i$ is a one electron hamiltonian
containing the eltronic kinetic energy and the nucleus-electron coulomb attraction and
$\hat{J}_i\ \text{and}\ \hat{K}_i$ are the coulomb integral operator and the exchange integral operator
respectively \cite{Jensen:1999, Cramer:2004} which apply the two-electron integrals of the same name on a
oribtal $\phi_j$.

\subsubsection{Variational principle on a Slater determinant}

The reason for writting the energy of a Slater Determinant as we did above in \ref{eq:SD}
is so that we could apply the variational principle on the wave function. We
construct a Lagrange equation as in \cite{Jensen:1999}:
\begin{align}
  L &= E - \sumdef{ij}\lambda_{ij}\left(\braket{\phi_i|\phi_j} - \delta_{ij} \right) \\
  \delta L &= \delta E - \sumdef[n]{ij} \lambda_{ij}(\braket{\delta\phi_i|\phi_j} + \braket{\phi_i|\delta\phi_j})\label{eq:dlagrange}
\end{align}
where
\begin{align}\label{eq:dESD}
  \begin{split}
    \delta E &= \sumdef[n]{i}\left(\braket{\delta\phi_i|\hat{h}|\phi_i} + \braket{\phi_i|\hat{h}|\delta\phi_i}\right)\\
    &+ \frac{1}{2}\sumdef[n]{ij}\left(\braket{\delta\phi_i|\hat{J}_j - \hat{K}_j|\phi_i} + \braket{\phi_i|\hat{J}_j - \hat{K}_j|\delta\phi_i}\right)\\
    &+ \frac{1}{2}\sumdef[n]{ij}\left(\braket{\delta\phi_j|\hat{J}_i - \hat{K}_i|\phi_j} + \braket{\phi_j|\hat{J}_i - \hat{K}_i|\delta\phi_j}\right)
  \end{split}
\end{align}
and  $\lambda$ is the Lagrange multiplier.
The goal is to find a set of orbitals that give us a minimum of the lagrange
equation $\delta L = 0$.
\subsubsection{Hartree-Fock equations}
We can remove the multiplication by a half outside the sums as the third and fourth
terms count the same operations, and so do the fourth and sixth terms. We can also
substitute the Fock operator $\hat{F}$ into equation \ref{eq:dESD} as \cite{Jensen:2017}
\begin{align}\label{eq:dEfock}
  \delta E = \sumdef[n]{j}\left(\left\langle\delta\phi_i \left| \hat{F} \right| \phi_i \right\rangle + \left\langle\phi_i \left| \hat{F} \right| \delta\phi_i \right\rangle\right)\\\label{}
    \hat{F} = \hat{h} + \sumdef[n]{j}\left(\hat{J}_j - \hat{K}_j\right)
\end{align}

We substituate \ref{eq:dEfock} into \ref{eq:dESD} and use properties of complex conjugates
to manipulate the equation as shown in \cite{Jensen:2017} to give us the Hartree-fock
equations
\begin{equation}
\hat{F}\phi_i = \sumdef[n]{j}\lambda_{ij}\phi_j
\end{equation}
where we can construct a $n\times n$ matrix out of these sums called the Fock matrix.
We then diagonalize the Fock matrix and get the following set of \eival equations
\begin{equation}
\hat{F}\phi_i = epsilon_i\phi_i
\end{equation}
where the $\epsilon_i$ is the energy of each orbital. When diagonalizing the matrix the
orbitals get changed slightly, creating new orbitals. We denote these new orbitals with a $^\prime$
superscript.

In order for an orbital to be known, one must know the rest of the
orbitals. We solve this by iteratively forming the Fock matrix, diagonalizing it
and using these new orbitals to construct a new Fock matrix and repeate the
process until the change in the orbitals reach below a predetermined threshold
\cite{Helgaker:2012, Cramer:2004}. This iterative method is called \ac{SCF}.

The sum of the orbital energies do not give us the total sum of the system.
Since, for each electron,  we are accounting for the repulsion and exchange interactions with all the
other electrons, we end up counting twice, which means we have to substract half of
these interactions from the total orbital energy. This gives the total energy $E$ as
\cite{Jensen:2017}
\begin{equation}
  E = \sumdef[n]{i}\epsilon_i - \frac{1}{2}\sumdef[n]{ij}\left(\hat{J}_{ij} - \hat{K}_{ij}\right)
\end{equation}

\subsection{Density Functional Theory}
The method outlined above employed an iterative procedure to calculate the coefficients
a a linear combination of basis functions describing the wave function of the
system, which is then used to calculate the energy of the system. We see that we
still need to calculate the repulsion potential between every electron. This is
simplified by using a density matrix to weight each of the basis that make up the
wave function of the other electrons.

Using a slightly different method, we calculate the energy directly as a functional
of the electron density distribution \cite{Sorland, Cramer:2004} described by
\begin{equation}\label{eq:densityintegral}
  \rho(\rvec) = n \int\abs{\Psi_e(\rvec, \rvec_1, \rvec_2,..., \rvec_n)} \text{d}\rvec_1\text{d}\rvec_2...\text{d}\rvec_n
\end{equation}
which is an integral over all the possible configurations of the electrons. The
coefficient $n$ is the total amount of electrons on the system. This method,
\ac{DFT}, allows to discard computing the wave function in the iterative process and
simply use the density instead.

The electron density has properties that are characteristic to each system, the
molecule,  it is constructed from. The first is that its integral over all space
is equal to the amount of electrons $n$ on the system \cite{Cramer:2004}
\begin{equation}
  \int_{\Real^3}\rho \text{d}\rvec = n
\end{equation}
The second is that the density has maxima on the positions of the nuclei, and
these maxima have the following value dependent on the nuclear charge of said
nuclei \cite{Cramer:2004}.
\begin{equation}
 \diff{\rvec}\rho(\rvec)\Bigg|_{\rvec = \Rvec_I} = -2Z_I\rho(\rvec)
\end{equation}

The energy is determined by a functional $E[\rho]$ made up of functionals representing
the kinetic energy of the electrons $T[\rho]$, the attraction potential $V_{Ne}[\rho]$,
and the two electron repulsion $V_{e}[\rho]$ \cite{Sorland, PhysRev.136.B864}
\begin{align}
  E[\rho] &= V[\rho] + T[\rho] + V_{e}[\rho]\\\label{eq:energyfunctional}
  T[\rho] &\equiv\braket{\rho(\rvec)| \frac{1}{2}\nabla^2| \rho(\rvec)} \\
  V_{e}[\rho] &\equiv \frac{1}{2}\braket{v_{e}(\rvec)| \rho(\rvec)}\\
  V_{Ne}[\rho] &= \braket{v_{Ne}(\rvec)|\rho(\rvec)}
\end{align}
Where the potential functions $v_{Ne}(\rvec)$  and $v_{e}(\rvec)$ are defined as the Coulomb
attraction between the electrons and the nuclei and the coulomb repulsion of the
density with itself \cite{Sorland} respectively.

The functional in Equation \ref{eq:energyfunctional} is not an appropiate approximation
of the energy, as we are assumming in the construction of $T[\rho]$ and $V_{e}[\rho]$
that the energy is known for the individual parts of the disconnected system \cite{Sorland}.
This means we are missing the exhchange-correlation contributions of the energy $E_{xc}{\rho}$.
With that in mind, the proper energy functional is shown as follows
\begin{equation}\label{eq:adjustedenergyfunc}
  E[\rho] = V[\rho] + T[\rho] + V_{e}[\rho] + E_{xc}[\rho]
\end{equation}

By the variational principle discussed previously, we are trying to minimize the energy
to get the best representation of the system with the given basis. Here we try to
find minima by differentiating \ref{eq:adjustedenergyfunc} with respect to $\rho$ giving us
\begin{align}\label{eq:energyderiv}
   \diff{\rho}E[\rho] = \diff{\rho}\left( V[\rho] + T[\rho] + V_{e}[\rho]\right) + V_{xc}
\end{align}
Where $V_{xc}$ is the functional derivative $\diff{\rho}E_{xc}[\rho]$ \cite{Cramer:2004}.
We can see that $V_{xc}$ seems to be a correction to the gradient to our first definition
of the the energy \ref{eq:energyfunctional}. The system described by Equation
\ref{eq:energyfunctional} is a system of non-interacting electrons. In this
system we can define one-electron operators $\hat{h}_i^{KS}$ as
\begin{align}
  \hat{h}_i^{KS} \phi_i &= \epsilon_i \phi_i \\
  \hat{h}_i^{KS} &= - \frac{1}{2}\nabla^2_i + v_{Ne} + v_e(\rho) + V_{xc}
\end{align}
where $\phi$ are basis functions describing the function $\Psi$ from \ref{eq:densityintegral}
We can thus use $V_{xc}$ to iteratively minimize the energy in a systematic manner so we can
get the correct density for the system of interacting electrons \cite{Cramer:2004}.

This tells us that, given a set of basis functions $\phi$ describing
$\rho$ as defined in Equation \ref{eq:densityintegral}, we can compute the following
Kohn-Sham  matrix elements $\bar{K}_{\mu\nu}$ for a secular equation resembling that of the \ac{HF}
method. In this secular equation the elements $\bar{F}_{\mu\nu}$ are replaced by $\bar{K}_{\mu\nu}$.
\begin{equation}
  \bar{K}_{\mu\nu} = \braket{\phi_\mu|\left[- \frac{1}{2}\nabla^2_i + v_{Ne} + v_e(\rho) + V_{xc}\right]|\phi_\nu}
\end{equation}
The rest of the procedure follows the same process as the \ac{SCF} explained on
the previous section.
What remains is to define $V_{xc}$ which can be represented in several ways by different methods
based on the basis functions used to construct the density, the reader is directed to
read on this in \cite{Cramer:2004} as this thesis will not delve deeper on the
matter.

\begin{acronym}
\acro{AUS}[\href{https://www.sigma2.no/content/advanced-user-support}{AUS}]{Numerical Methods in Quantum Chemistry}
\acro{BO}{Born-Oppenheimer}
\acro{CTCC}[\href{http://www.ctcc.no}{CTCC}]{Centre for Theoretical and Computational Chemistry}
\acro{DC}{Dielectric Continuum}
\acro{DFT}{Density Functional Theory}
\acro{EFP}{Effective Fragment Potential}
\acro{EU}{European Union}
\acro{HF}{Hartree-Fock}
\acro{Hylleraas}[\href{https://www.mn.uio.no/hylleraas/english/}{Hylleraas}]{Hylleraas
  Centre for Quantum Molecular Sciences}
\acro{HPC}{High Performance Computing}
\acro{KTH}{Royal Institute of Technology}
\acro{LDA}{Local Density Approximation}
\acro{MCD}{Magnetic Circular Dichroism}
\acro{MCSCF}{Multiconfiguration Self Consistent Field}
\acro{MM}{Molecular Mechanics}
\acro{MW}{Multiwavelet}
\acro{NFR}{Norwegian Research Council}
\acro{NMQC}[\href{http://www.ctcc.no/events/conferences/2015/numeric-conference/}{NMQC}]{Numerical Methods in Quantum Chemistry}
\acro{NOTUR}[\href{https://www.notur.no/}{NOTUR}]{Norwegian Metacenter for Computational Science}
\acro{PCM}{Polarizable Continuum Model}
\acro{PI}{Primcipal Investigator}
\acro{QC}{Quantum Chemistry}
\acro{QM}{Quantum Mechanics}
\acro{QM/MM}{Quantum Mechanics/Molecular Mechanics}
\acro{ROA}{Raman Optical Activity}
\acro{SC}{semiconductor}
\acro{SCF}{Self Consistent Field}
\acro{SHG}{Second Harmonic Genertation}
\acro{STSM}{Short-term scientific mission}
\acro{TPA}{Two-Photon Absorption}
\acro{WP}{Work Package}
\acro{CBS}{Complete Basis Set}
\acro{TCG}{Theoretical Chemistry Group}
\acro{vdW}{van der Waals}
\acro{SE}{Schrödinger Equation}
\acro{PES}{Potential Energy Surface}
\acro{LCAO}{Linear Combination of Atomic Orbitals}
\acro{MRA}{Multi-Resolution Analysis}
\acro{NS}{Nonstandard}
\end{acronym}

\biblio
\end{document}

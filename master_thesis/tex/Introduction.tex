\makeatletter
\def\input@path{{../}}
\makeatother
\documentclass[../master_thesis.tex]{subfiles}
\begin{document}
\chapter{Introduction}
\section{Background}
Gas phase calculations for Quantum chemical systems are well defined and
reliable results are able to be calculated from them, but it is rare for a chemist
to work with exclusively gas phase properties. Most of the processes of interest
in chemical, biological and physical mechanisms happen while in a solvent \cite{FossoTande:2013ka}.
Therefore it is important to be able to produce reliable results when working
with these systems. Many models exist to solve this system, outlined in chapter
\ref{chap:Solvent_effect}, but the family of models we are focusing on is \ac{PCM}.

A big hurdle of \ac{PCM} is working with the sharp transition between solvent and
solute. Numerous methods have been implemented in order to minimize the discontinuity
at this boundary \cite{Tomasi:2005ipa}. Our approach is to define both the solvent
and substrate environments as an analytical function represented with \ac{MW} basis.

\ac{MW} are a part of a family of basis called \ac{MRA} bases. These basis have useful
characteristics for development of solutions of partial differential equations.
These are orthogonality of wavelet spaces, vanishing moments and locally adaptive
representations of functions, which allow for further refining critical areas of a system.


\section{Summary of Results}

We found that the implementation managed to produce values that did not differ
significantly from exact calculations of the properties tested, these being the Born energy of
the solvent \cite{Tomasi:1994wt} and Gauss' theorem as defined in \cite{Sorland}.
The Energy calculations yielded results that converged towards Gaussian results
as the radii increased. We also tested the Variational formulation of the
\ac{GPE} shown in \cite{Lipparini:2010bg}, which gave promising results for
small systems, such as \ce{Li^+}, but showed room for improvement for bigger systems.

\section{Structure of thesis}
In chapter \ref{chap:Quantum_chemistry} of this thesis we will present the quantum
and computational Chemical formalism and methods used among these is the
Born--Oppenheimer approximation, variational principle, \ac{HF} method and
\ac{DFT}.

basis for solving quantum systems. Some examples of Wavelet basis are shown and
Chapter \ref{chap:MW_basis} will mention some basis sets and define \ac{MW}
the implementation of different operators is described.

Chapter \ref{chap:Solvent_effect}
describes the particular system we work with in this thesis, namely, Solvent
effect and Reaction Potential. This chapter will also introduce a new variational
formulation of the Generalized Poisson equation used in solving the electrostatic
Reaction Field problem.

Chapter \ref{chap:implementation} describes how the Reaction Field Potential was implemented
using \mrchem. It will also describe the way the variational formulations was implemented.

Chapter \ref{chap:Results} Goes through the systems that were used to test the implementation,
along with the type of tests that were performed. The results will be tabulated and
some figures will be shown. Finally an analysis and discussion of the results will be
presented along with points of improvement and future development.
\section{Notation}
Notation differs slightly between works in computational and Quantum Chemistry.
Most of the equations and formulas use the same standard notation, while others
might differ slightly in accents and other details in their notations. Therefore
in this section I will summarize the notation used in this thesis.

The following Table \ref{tab:notation} shows most of the notation used for
common objects in mathematics and quantum mechanics.

\begin{table}[h!]

    \resizebox{\textwidth}{!}{
  \begin{tabular}{|l|l|c|}
    \hline
    Object & Description of notation & Examples\\  \hline
    \rule{0pt}{12.5pt}vectors & lower case letter, or symbol, with an arrow above them & $\rvec, \vec{x}, \vec{\Psi}$ \\  \hline
    \rule{0pt}{12.5pt}vector elements & lower case letter, or symbol, with an arrow above them and index subscripts &  $\rvec_i, \vec{x}_1, \vec{\Psi}_x$ \\ \hline
    \rule{0pt}{12.5pt}matrices & Upper case letter or symbols with a line above them &  $\bar{A}, \bar{F}, \bar{S}$ \\ \hline
    \rule{0pt}{12.5pt}matrix elements &   Upper case letter  with a line above them and index subscripts (rows, columns) &  $\bar{A}_{ij}, \bar{F}_{11}, \bar{S}_{\mu\nu}$\\  \hline
    \rule{0pt}{12.5pt}Operators & Presented as a symbol with a \^~above it & $\hat{O}, \hat{H}, \hat{F}$  \\  \hline
    \rule{0pt}{12.5pt}functions & presented by a letter with their input variables enclosed in brackets () & $f(x), \Psi(\rvec)$\\  \hline
    \rule{0pt}{12.5pt}functionals & presented by a letter with their input functions enclosed in square brackets [] & $E[f], F[\Psi]  $\\  \hline
  \end{tabular}}{\caption{Notation used in this thesis}\label{tab:notation}}
\end{table}

Additionally we will mostly be working with atomic units. This means that many common
constants are reduced to unit value. The constants used in this thesis are the
elementary charge $e$ the reduced Planck constant $\hbar$, the electron mass $m_e$ and
vacuum permittivity $4\pi\epso$ are all unit value.

Finally, we will be working in Dirac notation. This means that we write a state
vector of a system described by wave function $\Psi$ as $\ket{\Psi}$ and its
complex conjugate as $\bra{\Psi}$. The scalar product in Dirac notation is
defined as
\begin{equation}
  \braket{\Psi|\Phi} = \int_{\Real^3} \Psi^{\star}\Phi \text{d}\rvec
\end{equation}
\begin{acronym}
\acro{AUS}[\href{https://www.sigma2.no/content/advanced-user-support}{AUS}]{Numerical Methods in Quantum Chemistry}
\acro{BO}{Born-Oppenheimer}
\acro{CTCC}[\href{http://www.ctcc.no}{CTCC}]{Centre for Theoretical and Computational Chemistry}
\acro{DC}{Dielectric Continuum}
\acro{DFT}{Density Functional Theory}
\acro{EFP}{Effective Fragment Potential}
\acro{EU}{European Union}
\acro{HF}{Hartree-Fock}
\acro{Hylleraas}[\href{https://www.mn.uio.no/hylleraas/english/}{Hylleraas}]{Hylleraas
  Centre for Quantum Molecular Sciences}
\acro{HPC}{High Performance Computing}
\acro{KTH}{Royal Institute of Technology}
\acro{LDA}{Local Density Approximation}
\acro{MCD}{Magnetic Circular Dichroism}
\acro{MCSCF}{Multiconfiguration Self Consistent Field}
\acro{MM}{Molecular Mechanics}
\acro{MW}{Multiwavelet}
\acro{NFR}{Norwegian Research Council}
\acro{NMQC}[\href{http://www.ctcc.no/events/conferences/2015/numeric-conference/}{NMQC}]{Numerical Methods in Quantum Chemistry}
\acro{NOTUR}[\href{https://www.notur.no/}{NOTUR}]{Norwegian Metacenter for Computational Science}
\acro{PCM}{Polarizable Continuum Model}
\acro{PI}{Primcipal Investigator}
\acro{QC}{Quantum Chemistry}
\acro{QM}{Quantum Mechanics}
\acro{QM/MM}{Quantum Mechanics/Molecular Mechanics}
\acro{ROA}{Raman Optical Activity}
\acro{SC}{semiconductor}
\acro{SCF}{Self Consistent Field}
\acro{SHG}{Second Harmonic Genertation}
\acro{STSM}{Short-term scientific mission}
\acro{TPA}{Two-Photon Absorption}
\acro{WP}{Work Package}
\acro{CBS}{Complete Basis Set}
\acro{TCG}{Theoretical Chemistry Group}
\acro{vdW}{van der Waals}
\acro{SE}{Schrödinger Equation}
\acro{PES}{Potential Energy Surface}
\acro{LCAO}{Linear Combination of Atomic Orbitals}
\acro{MRA}{Multi-Resolution Analysis}
\acro{NS}{Nonstandard}
\end{acronym}

\biblio
\end{document}

\makeatletter
\def\input@path{{../}}
\makeatother
\documentclass[../master_thesis.tex]{subfiles}
\begin{document}
\chapter{Introduction}
\section{Background}

\section{Summary of Results}

\section{Structure of thesis}
In chapter \ref{chap:Quantum_chemistry} of this thesis we will present the quantum
and computational Chemical formalisms and methods used Among these is the
Born-oppenheimer approximation, Variational principle, Hartree-Fock method and
Density Functional Theory.

basis for solving quantum systems. Some examples of Wavelet basis are shown and
Chapter \ref{chap:MW_basis} will mention some basis sets and define multiwavelet
the implementation of different operators is described.

Chapter \ref{chap:Solvent_effect}
describes the particular system we work with in this thesis, namely, Solvent
effect and Reaction Potential. This chapter will also introduce a new variational
formulation of the Generalized poisson equation used in solving the electrostatic
Reaction Field problem.

Chapter \ref{chap:implementation} describes how the Reaction Field Potential was implemented
using \mrchem. It will also describe the way the variational formulatins was implemented.

Chapter \ref{chap:Results} Goes through the systems that were used to test the implementation,
along with the type of tests that were perfomed. The results will be tabulated and
some figures will be shown. Finally an analysis and discussion of the results will be
presented along with points of improvement and future development.
\section{Notation}
Notation differs slightly between works in computational and Quantum Chemistry.
Most of the equations and formulas use the same standard notation, while others
might differ slightly in accents and other details in their notations. therefore
in this section i will sumarize the notation used in this thesis.

The following Table \ref{tab:notation} shows most of the notation used for
common objects in mathematics and quantum mechanics.

\begin{table}[h!]

    \resizebox{\textwidth}{!}{
  \begin{tabular}{|l|l|c|}
    \hline
    Object & Description of notation & Exampls\\  \hline
    \rule{0pt}{12.5pt}vectors & lower case letter, or symbol, with an arrow above them & $\rvec, \vec{x}, \vec{\Psi}$ \\  \hline
    \rule{0pt}{12.5pt}vector elements & lower case letter, or symbol, with an arrow above them and index subscripts &  $\rvec_i, \vec{x}_1, \vec{\Psi}_x$ \\ \hline
    \rule{0pt}{12.5pt}matrices & Upper case letter or symbols with a line above them &  $\bar{A}, \bar{F}, \bar{S}$ \\ \hline
    \rule{0pt}{12.5pt}matrix elements &   Upper case letter  with a line above them and index subscripts (rows, columns) &  $\bar{A}_{ij}, \bar{F}_{11}, \bar{S}_{\mu\nu}$\\  \hline
    \rule{0pt}{12.5pt}Operators & Presented as a symbol with a \^~above it & $\hat{O}, \hat{H}, \hat{F}$  \\  \hline
    \rule{0pt}{12.5pt}functions & presented by a letter with their input variables enclosed in brackets () & $f(x), \Psi(\rvec)$\\  \hline
    \rule{0pt}{12.5pt}functionals & presented by a letter with their input functions enclosed in square brackets [] & $E[f], F[\Psi]  $\\  \hline
  \end{tabular}}{\caption{}\label{tab:notation}}
\end{table}

Additionally we will mostly be working with atomic units. This means that many common
constants are reduced to unit value. The constants used in this thesis are the
elementary charge $e$ the reduced Planck constant $\hbar$, the electron mass $m_e$ and
vacuum premitivity $4\pi\epso$ are all unit value.

Finally, we will be working in Dirac notation. This means that we write a state
vector of a system described by wave function $\Psi$ as $\ket{\Psi}$ and its
complex conjugate as $\bra{\Psi}$. The scalar product in Dirac notation is
defined as
\begin{equation}
  \braket{\Psi|\Phi} = \int_{\Real^3} \Psi^{\star}\Phi \text{d}\rvec
\end{equation}

\biblio
\end{document}

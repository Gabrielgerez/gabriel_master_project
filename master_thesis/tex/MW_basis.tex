\makeatletter
\def\input@path{{../}}
\makeatother
\documentclass[../master_thesis.tex]{subfiles}
\begin{document}
\chapter{\ac{MW} Basis}
As stated on the previous chapter, the main goal of computational chemistry is
to approximate systems in order to calculate their energy through the \ac{SE}. %cite please
These systems are completely described by wave functions \cite{Cohen:1973}.
This means that to approximate a solution to the system is to approximate its
wave function.

In order to construct a solution (wave function) to the \ac{SE} of a given
system one uses sets of orthogonal functions with differing properties. These
sets of functions construct a basis for the space on which the wave functions
are projected into. These sets are thus called basis sets \cite{Cramer:2004}
and they are essential to solving many body systems. In this text we will be
focusing in the \ac{MW} basis from \ac{MRA} methods.

\section{\ac{MRA}}
\subsection{Definition}
Consider that we have a function $\varphi \in L^2(\Real)$ where its translations
and dilations are described as \cite{Schneider:2007}
\begin{equation}
  \varphi^j_k(x) = 2^{\frac{j}{2}}\varphi(2^jx - k),\  j,k \in \mathbb{Z}
\end{equation}.
And the function $\varphi(x)$ satisfies the two-scale difference relations \cite{Beylkin:MRA, Schneider:2007, Sorland}:
\begin{align}
  \begin{split}
    \varphi(x) &= \varphi(2x) + \varphi(2x - 1)\\
    \varphi^j_k(x) &= \varphi^{j+1}_{2k}(2^{j+1}x - 2k) + \varphi^{j+1}_{2k+1}(2^{j+1}x - 2k - 1)
  \end{split}
\end{align}
Where $j$ is the scale of the function and $k$ is the translation of the function
\cite{Sorland}. A normalization constant is included in the definition of $\varphi$.
A space $V^n$ is spanned by translations of $\varphi_{nk}$.This space forms a
hierarchical chain of Linear
subspaces \cite{Beylkin:MRA}:
\begin{equation}
  V^0 \subset V^1 \subset ... \subset V^j \subset ... \subset L^2(\Real)\label{eq:seqsubspace}
\end{equation}
Where $V^0$ is spanned only by $\varphi_{0,0}(x)=\varphi(x)$ \cite{Sorland}.

If relation \ref{eq:seqsubspace} and the following refinement equation holds for $\varphi_{j,k}(x)$
one can call the subspaces $V^n$ or the functions $\varphi_{j,k}(x)$ build a \ac{MRA} of $L_2(\Real)$.
\begin{equation}
\varphi^j_k(x) = \sum_{k\in\mathbb{Z}} h^{j+1}_k\varphi^{j+1}_k(x)
\end{equation}
Where $h$ is a coefficient characteristic to the transformation between scales.
\subsection{The Haar Wavelet \ac{MRA}}
Following from now we will work with the Haar wavelet basis for simplicity \cite{Beylkin:MRA}.
Lets define the Haar function \cite{Schneider:2007} as
\begin{equation}
  \varphi^0_0 = \varphi(x) =
  \begin{cases}
  1 & \text{for} \ x\in [0,1)\\
  0 & \text{elsewhere}
\end{cases}
\end{equation}
Let us now define a second set of subspaces $W^n$. These are the orthogonal complements of $V^n$ \cite{Alpert1993}, also called difference subspaces,
defined as \cite{Beylkin:MRA, Sorland, Alpert1993}.
\begin{equation}
  W^n \oplus V^n = V^{n + 1} \label{eq:diffsubspace}
\end{equation}
The subspaces $W^n$ are then spanned by a set of functions defined by the translations and
dilations of $\psi(x)$:
\begin{equation}
  \psi_k^j(x) = 2^{\frac{j}{2}}\psi(2^jx - k),\  j,k \in \mathbb{Z} \label{eq:haarwavelet}
\end{equation}
Where $\psi(x)$ is called the Haar wavelet \cite{Schneider:2007} and is defined as:
\begin{equation}
  \psi^0_0 = \psi(x) =
  \begin{cases}
  1 & \text{for} \ x\in [0,\frac{1}{2})\\
  -1 & \text{for}\ x\in [\frac{1}{2}, 1)\\
  0 & \text{elsewhere}
\end{cases}
\end{equation}
And $ \varphi$ is related to $\psi$ by the following two-scale difference relation \cite{Beylkin:MRA, Schneider:2007, Sorland}:
\begin{align}
  \begin{split}\label{eq:2scalewavelet}
    \psi(x) &= \varphi(2x) - \varphi(2x - 1)\\
    \psi^j_k(x) &= \psi^{j+1}_{2k}(2^{j+1}x - 2k) + \psi^{j+1}_{2k+1}(2^{j+1}x - 2k - 1)
  \end{split}
\end{align}
The functions $\varphi^j_k$ and $\psi^j_k$ are orthonormal
and dense \cite{Beylkin:MRA, Sorland, SRJensen:2014} in $L^2(\Real)$.

The Definition on Equation \ref{eq:diffsubspace} can be applied recursively in order to
get any space $V^n$ as long as one knows the first subspace $V^0$ and one has a method for constructing the
subspace $W^m$ from $V^0$ and $W^{m-1}$.
\begin{equation}
    V^0 \oplus W^0 \oplus W^1 \oplus ... \oplus W^{n-1}  = V^n \label{eq:recursivespace}
\end{equation}

Projecting a function $f(x)$ unto this basis would be then a weighted linear combination
of the Haar functions, but taking into account the definition on Equation \ref{eq:recursivespace} one arrives
at \cite{Sorland}.
\begin{equation}\label{eq:projectftohaar}
  f(x)\approx \sum^{2^j -1}_k s^j_k\varphi^j_k = s^0_0\varphi^0_0 + \sum^{N - 1}_j\sum^{2^j -1}_k d^j_k\psi^j_k
\end{equation}
where $d$ are the difference coefficients and $s$ are the scaled averages of dyadic intervals of the function $f(x)$

The scaling coefficients $s^j_k$ are computed by the projection $\braket{\varphi^j_k(x)|f(x)}$.
Likewise the difference coefficients $d^j_k(x)$ are computed by the projection \newline$\braket{\psi^j_k(x)|f(x)}$.
Because of the way the Haar function is defined, we can define of the scaling coefficients as
scaled averages of $f(x)$ at intervals $2^{-j}$ \cite{Sorland, Beylkin:MRA}
\begin{equation}
  s^j_k = \int_{\Real}\varphi^j_k(x)f(x)\text{d}x = \int^{2^{-j}(k + 1)}_{2^{-j}k} f(x) \text{d}x\label{eq:scalecoeff1}
\end{equation}
The subsequent scaling coefficients can be obtained as
\begin{align}
  \begin{split}\label{eq:scalecoeffint}
    s^{j}_k &= \int_{\Real}\varphi^{j-1}_k(x)\text{d}x\\
    s^{j}_k &= 2^{\frac{j}{2}}\int_{\Real}\varphi(2^{j}x - k)f(x)\text{d}x\\
    s^{j}_k &= 2^{\frac{j}{2}}\int_{2^{-j}(k-1)}^{2^{-j}k}f(x)\text{d}x\\
  \end{split}
\end{align}
We can then obtain the difference coefficients by using Equations \ref{eq:scalecoeff1}, \ref{eq:haarwavelet} and \ref{eq:2scalewavelet}:
\begin{align}
  \begin{split}\label{eq:diffcoeffint}
    d^{j - 1}_k &= \int_{\Real}\psi^{j-1}_k(x)f(x)\text{d}x\\
    d^{j - 1}_k &= 2^{\frac{j - 1}{2}}\int_{\Real}\psi(2^{j-1}x - k)f(x)\text{d}x\\
    d^{j - 1}_k &= 2^{\frac{j - 1}{2}}\left(\int_{\Real}\varphi(2^jx - 2k)f(x)\text{d}x  - \int_{\Real}\varphi(2^jx - 2k - 1)f(x)\text{d}x\right)\\
    d^{j - 1}_k &= 2^{\frac{j - 1}{2}}\left( \int^{2^{-j}(2k+1)}_{2^{-j}2k}f(x)\text{d}x - \int^{2^{-j}(2k+2)}_{2^{-j}(2k + 1)}f(x)\text{d}x \right)\\
    d^{j - 1}_k &= \frac{1}{\sqrt{2}}\left(s^{j}_{2k} - s^{j}_{2k+1} \right)
  \end{split}
\end{align}

The result of Equations \ref{eq:diffcoeffint} and \ref{eq:scalecoeffint} show us
that we can represent the projection of coefficients unto a coarser scale as an
orthogonal matrix \cite{Sorland, Beylkin:MRA}:
\begin{equation}
  \begin{pmatrix}
    d^{j}_k \\
    s^{j}_k
  \end{pmatrix} =
  \begin{pmatrix}
    \frac{1}{\sqrt{2}} & -\frac{1}{\sqrt{2}} \\
    \frac{1}{\sqrt{2}} & \frac{1}{\sqrt{2}}
  \end{pmatrix}
  \begin{pmatrix}
    s^{j+1}_{2k} \\
    s^{j+1}_{2k+1}
  \end{pmatrix}
\end{equation}
Projecting the coefficients into a more refined scale is just a transpose of the
above matrix:
\begin{equation}
  \begin{pmatrix}
    s^{j+1}_{2k} \\
    s^{j+1}_{2k+1}
  \end{pmatrix} =
  \begin{pmatrix}
    \frac{1}{\sqrt{2}} & \frac{1}{\sqrt{2}} \\
    -\frac{1}{\sqrt{2}} & \frac{1}{\sqrt{2}}
  \end{pmatrix}
  \begin{pmatrix}
    d^{j}_k \\
    s^{j}_k
  \end{pmatrix}
\end{equation}
\subsection{Projecting a gaussian function example}
As an example, lets approximate the function
$f(x) = \frac{10}{\sqrt{\pi}}e^{-100(x - 0.5)^2}$ in $L^2(\Real)$
with Haar basis up to scale $5$ using \ref{eq:projectftohaar}. Gives us the
following figures.
%put the plots here

\section{\ac{MW} \ac{MRA}}
%weakness of wavelet that might be fixed with multiwavelet
\subsection{Constructing the basis functions in one dimension}
Following the same basics as in the Haar basis from the previous section we can
define a hierarchical set of multi-resolution spaces $V^j_l$ where \cite{Frediani:2013}:
\begin{align}
  \begin{split}
    V^j_l \defeq \{&f: \text{all polynomials of degree} \leqslant
    l\  \\
    &\text{on}\  (2^{-j}k,2^{-j}(k+1))\ \text{for}\ 0\leqslant k < 2^{n},\\
    &f\  \text{vanishes elsewhere}\}
  \end{split}
\end{align}
and
\begin{equation}
  V^0_l \subset V^1_l \subset ... \subset V_l^j \subset ... \subset L^2(\Real)
\end{equation}

We again define subspaces $W^^l$ as the orthogonal complements of $V^l_j$
\cite{Alpert1993} defined as:
\begin{equation}
  V_l^j \oplus W^j_l = V_l^{j+1}
\end{equation}
 with orthogonal basis functions  $\psi^j_{lk}$ which are translations and dilations of functions $\psi_i$:
 \begin{equation}
   \psi^j_{ik}(x) = 2^\frac{j}{2}\psi_i(2^jx-k), \ i=1, ...,l;\ k \in \mathbb{Z}\label{eq:mwbasisfuncs}
 \end{equation}
 The functions $\psi_1, ...,\psi_l$ are piecewise polynomial, orthogonal to lower order polynomials and
 vanish outside $[0,1]$ and their subscripts denote the polynomial order \cite{Alpert1993}
\begin{align}
  \int_0^1\psi_i(x)x^m \text{d}x = 0,\ m = 0, 1, ..., l-1
\end{align}
We now have defined certain characteristics for these scaling functions $\psi_i(x)$,
what remains is to construct appropriate functions that fit these parameters.
We introduce a set generic functions $f_i$ that will be used as starting points
to construct the scaling functions $\psi_1, ...,\psi_l$. The following restrictions must be
fulfilled in the definition of the trial functions $f_i$ so that they can be
used to construct $\psi_1, ...,\psi_l$.
\cite{Alpert1993, Beylkin:MRA}:
\begin{enumerate}
  \item functions $f_i$ on the interval $(0,1)$ are polynomials of degree $l-1.$
  \item functions $f_i$ extend to the interval $(-1, 0)$ as even or odd functions
  dependent to the parity $i + l - 1$.
  \item functions $f_1,...,f_l$ are orthonormal with respect to each other $\braket{f_i| f_j} = \delta_{ij}, \ i, i = 1, ..., l.$
  \item functions $f_j$ have vanishing moments, $$\int_{-1}^1f_j(x)x^i\text{d}x = 0, \ i = 0, 1, ..., j+l-2.$$ Which means that they
  are orthogonal to lower order polynomials, such that they do not have overlap.
\end{enumerate}
The functions $f^{(1)}_1, f^{(1)}_2, ..., f^{(1)}_l$ (where the superscript
$(1)$ refer to the step in the construction)
must fulfill \cite{Alpert1993}:
\begin{equation}
  f^1_j(x) =
  \begin{cases}
  x^{j-1} & \text{for} \ x\in (0,1)\\
  -x^{j-1} & \text{for}\ x\in (-1, 0)\\
  0 & \text{elsewhere}.
\end{cases}
\end{equation}

The $2l$ functions $1, x,..., x^{l-1} $ and $ f^{(1)}_1, f^{(1)}_2, ..., f^{(1)}_l$ are linearly
independent by construction, thus they span the space of polynomial functions of
degree $< l$ on the intervals $(0, 1)$ and $(-1, 0)$ \cite{Alpert1993, Beylkin:MRA}. We do the
following process to find the functions $f_i$ \cite{Alpert1993}:
\begin{enumerate}
  \item Using the Gram-Schmidt process we orthogonalize $f^{(1)}_j$ with respect to $1, x,..., x^{l-1}$
  giving us the functions $f_j^{(2)}$ for $j = 1, ..., l$.
  \item  We try to find at least one $f^{(2)}_j$
  which is not orthogonal to $x^{l}$ and reorder the functions so that $\braket{f_1^{(2)}| x^{l}} \neq 0$.
  We now need to find an $a_j$ for the equation $f_j^{(3)} = f_j^{(2)}-a_j\cdot f_0^{(2)}$ such that
  $\braket{f_j^{(3)}| x^{l}} = 0\  \text{for}\  l = 2, ..., l$. We repeat this process orthogonalizing
  to $x^{k+1}, ..., x^{2k-2}$, each turn yielding $f_1^{(2)}, f_2^{(3)}, f_3^{(4)}, ..., f_k^{(k+1)}$ such that
  $\braket{f_j^{(j+1)}|x^i} = 0 \ \text{for}\  i\leqslant j+l-2$.
  \item  Lastly we perform the Gram-Schmidt orthogonalization process on
  $f_{l}^{l+1}, \\f_{l-1}^{l}, f_{l-2}^{l-1}, ..., f_{2}^{1}$ in that order to get
  $ f_l, f_{l-1}, ..., f_1$.
\end{enumerate}

We now construct the functions $h_l$ \cite{Alpert1993}:
\begin{equation}
  \psi_i(x)= \sqrt{2}f_i(2x-1), \ i= 1, ..., l
\end{equation}
which they themselves build the functions in Equation \ref{eq:mwbasisfuncs}
defining a basis for scaling subspace $W^^l$.

From the definition of these subspaces
we know that we need to define a basis for the subspace $V_l^0$ which is the
orthogonal complement of $W^0_l$ in $V_l^1$ \cite{Alpert1993}. The basis functions
$\phi_i(x)$ of the subspace $V_l^0$ can be defined using the following two-scale
difference equations \cite{Beylkin1999AdaptiveSO}, which are analogous to the
two-scale difference equations in \ref{eq:2scalewavelet}.
\begin{align}
  \psi_i(x) &= \sqrt{2}\sum^{l-1}_{j=0}\left(\bar{H}^{(0)}_{ij}\psi_j(2x) + \bar{H}^{(1)}_{ij}\psi_j(2x-1)\right), \ i = 0,...,l-1 \\
  \phi_i(x) &= \sqrt{2}\sum^{l-1}_{j=0}\left(\bar{G}^{(0)}_{ij}\psi_j(2x) + \bar{G}^{(1)}_{ij}\psi_j(2x-1)\right), \ i = 0,...,l-1
\end{align}
Where $ \bar{H} \ \text{and}\ \bar{G} $ are quadrature mirror filter matrices \cite{Beylkin1999AdaptiveSO} which have
the following properties:
\begin{align}
  \bar{H}^{(0)}\bar{H}^{(0)T} + \bar{H}^{(1)}\bar{H}^{(1)T} &= \bar{I} \\
  \bar{G}^{(0)}\bar{G}^{(0)T} + \bar{G}^{(1)}\bar{G}^{(1)T} &= \bar{I} \\
  \bar{H}^{(0)}\bar{G}^{(0)T} + \bar{H}^{(1)}\bar{G}^{(1)T} &= \bar{0}
\end{align}
Which can be summarized in the orthogonal block matrix $\bar{U}$ \cite{Beylkin1999AdaptiveSO}
\begin{equation}
  \bar{U} =
  \begin{pmatrix}
    \bar{H}^{(0)} & \bar{H}^{(1)} \\
    \bar{G}^{(0)} & \bar{G}^{(1)}
  \end{pmatrix}
\end{equation}
Which lets us put the two two-scale difference equations in the following way \cite{Sorland}
\begin{equation}
  \begin{pmatrix}
    \vec{\psi}(x) \\
    \vec{\phi}(x)
  \end{pmatrix}
  = \sqrt{2}
  \begin{pmatrix}
    \bar{H}^{(0)} & \bar{H}^{(1)} \\
    \bar{G}^{(0)} & \bar{G}^{(1)}
  \end{pmatrix}
  \begin{pmatrix}
    \vec{\psi}(2x) \\
    \vec{\psi}(2x-1)
  \end{pmatrix}
\end{equation}
and subsequently
\begin{equation}
  \begin{pmatrix}
    \vec{\psi}(2x) \\
    \vec{\psi}(2x-1)
  \end{pmatrix}
  = \frac{1}{\sqrt{2}}
  \begin{pmatrix}
    \bar{H}^{(0)} & \bar{G}^{(0)} \\
    \bar{H}^{(1)} & \bar{G}^{(1)}
  \end{pmatrix}
  \begin{pmatrix}
    \vec{\psi}(x) \\
    \vec{\phi}(x)
  \end{pmatrix}
\end{equation}
Where the function vectors $\vec{\phi}$ and $\vec{\psi}$ represent the set of
all $l-1$ functions $\phi_i$ and $\psi_i$ of order $i = 1, 2, ..., l$ respectively.

\subsection{Choices of Scaling functions }
Now that we have stated how to build the scaling and wavelet basis functions
we need to make a choice of functions $f$ to build said basis.
Here we will briefly show two examples of polynomial functions used to create the
basis. These two are the Legendre polynomials and the Lagrange interpolating
polynomials \cite{Beylkin:MRA, Beylkin1999AdaptiveSO}
\paragraph{Legendre basis}
The Legendre scaling function is defined as follows
\begin{align}
  \phi_j(x)
  \begin{cases}
    \sqrt{2j+1} P_j(2x-1), \ x&\in [0,1)\\
    0,\ x&\notin (0, 1)
  \end{cases}
\end{align}
Where $P_j$ are the Legendre polynomials of order $j$ defined in $[-1, 1]$ \cite{Beylkin:MRA}
Following are some examples of the polynomials together with figures of the first few terms
of the functions. These functions have the advantage of being simple to compute,
since each incremental polynomial order only adds a single term to the function.

\paragraph{Lagrange interpolating basis}
  \begin{align}
    \varphi_i &= \frac{1}{\sqrt{w_i}}l_i(x), \ i = 0, ..., M-1\\
    l_i(x) &= \prod^{M-1}_{k = 0, k\neq i} \frac{x-x_k}{x_i-x_k} \\
    w_i &= \frac{1}{M\hat{P}^\prime_M(2x_i-1)P_{M-1}(2x_i-1)}
  \end{align}
Where $M$ is the scale of the subspace, the $P$ functions are the Legendre polynomials and
$x_0, ..., x_{M-1}$ are the roots of $P_M(2x-1)$ \cite{Beylkin:MRA}.
These scaling functions have the characteristic of $l_i(x_j)=\delta_{ij}$ simplifying
integrals and thus projections of the basis.

Going from one dimension to $d$ dimensions is to use a tensor product method as shown in
\cite{Frediani:2013}.
\section{Operators}
\subsection{\ac{NS} form Representation of Operators}
Lets us define two projection operators $\hat{P}^k_n \ \text{and}\ \hat{Q}^k_n $  and their relation as:
\begin{align}
  \hat{P}^n &: \ L^2([0, 1]) \to V^n \\
  \hat{Q}^n &:\  L^2([0, 1]) \to W^n \\
  \hat{P}^{n+1} &= \hat{P}^n + \hat{Q}^n\\
  \hat{P}^n &= \hat{P}^0 + \hat{Q}^0 + \hat{Q}^1 + ...  \hat{Q}^{n-1}\label{eq:iterprojection}
\end{align}

A function $f$ would then be projected into a scale $n$ by applying the
operators as \cite{Frediani:2013}
\begin{equation}
   f^{(n)} = \hat{P}^n f
\end{equation}
Its projection unto a more refined scale would then be \cite{Frediani:2013}
\begin{align}\label{eq:refinef}
  \begin{split}
    f^ {(n+1)} = f^{(n)} + df^{(n)}\\
    df^{(n)} = \hat{Q}^nf
  \end{split}
\end{align}.
Following Equation \ref{eq:iterprojection} we can expand the equation above to
\begin{equation}
  f^{(N)} = f^{(0)} + df^{(0)} + df^{(1)} + ... + df^{(N-1)}
\end{equation}
Which, for a given refinement level $N$, is a good approximation of $f$.
\begin{equation}
  f \approx f^{(N)}
\end{equation}

The representation of a linear operator $\hat{T}$ onto a scale $n$ is
\begin{equation}
  \hat{T}^n = \hat{P}^n\hat{T}\hat{P}^n
\end{equation}
and unto a more refined scale as
\begin{align}\label{eq:refiningT}
  \begin{split}
    \hat{T}^{n+1} &= \hat{P}^{n+1}\hat{T}\hat{P}^{n+1}\\
    \hat{T}^{n+1} &= (\hat{P}^n + \hat{Q}^n)\hat{T}(\hat{P}^n + \hat{Q}^n)\\
    \hat{T}^{n+1} &= \hat{P}^n\hat{T}\hat{P}^n + \hat{P}^n\hat{T}\hat{Q}^n + \hat{Q}^n\hat{T}\hat{P}^n + \hat{Q}^n\hat{T}\hat{Q}^n
  \end{split}
\end{align}
We define a set of 3 operators which will help us describe the more refined operator
$\hat{T}$.
\begin{align}
  \hat{A}^n & \defeq\hat{Q}^n\hat{T}\hat{Q}^n: W^n \to W^n\\
  \hat{B}^n & \defeq\hat{Q}^n\hat{T}\hat{P}^n: V^n \to W^n\\
  \hat{C}^n & \defeq\hat{P}^n\hat{T}\hat{Q}^n: W^n \to V^n
\end{align}
So we can rewrite the last term in Equation \ref{eq:refiningT} as
\begin{equation}\label{eq:refineT}
  \hat{T}^{n+1} =\hat{A}^n + \hat{B}^n + \hat{C}^n + \hat{T}^n
\end{equation}
Repeating this iteratively we get:
\begin{equation}
  T^N = T^0 + \sum^N_{n=0}\left( \hat{A}^n + \hat{B}^n + \hat{C}^n\right)
\end{equation}
Which we can say, given a refinement level $N$ is a good approximation of $T$
\begin{equation}
  T \approx T^N
\end{equation}

Let us now define a function $g = \hat{T}f$ which is the resulting function from applying the
unprojected operator $\hat{T}$ unto the unprojected $f$. We want to represent this
operation in the \ac{MW} basis as
\begin{equation}
  g^{(n)} = \hat{P}^ng =  \hat{P}^n\left(\hat{T}f\right)
\end{equation}
but we do not know how $g$ looks like. We have the projected function $f^{ (n)}$ and the
representation of the operator $\hat{T}^n$. We do the following set of manipulations to
define a new projected function $\tilde{g}^{(n)}$ \cite{Frediani:2013}:
\begin{align}
  \begin{split}\label{eq:definegtilde}
    g^{(n)} = \hat{P}^ng &=  \hat{P}^n\left(\hat{T}f\right)\\
     &= \hat{P}^n\hat{T}\left(\hat{P}^n + 1 - \hat{P}^n\right)f\\
     &= \hat{P}^n\hat{T}\hat{P}^n\hat{P}^nf + \hat{P}^n\hat{T}\left(1 - \hat{P}^n\right)f\\
     &=   \hat{T}^nf^{(n)} + \hat{P}^n\hat{T}\left(1 - \hat{P}^n\right)f\\
    \tilde{g}^{(n)} &= \hat{T}^nf^{(n)}\\
  \end{split}
\end{align}
Where we know $ \hat{T}^n$ and $f^{(n)}$ as defined above and we focus only in
finding $\tilde{g}^{(n)}$. We substitute equations \ref{eq:refinef} and \ref{eq:refineT}
into the last term of equation \ref{eq:definegtilde} to get
\begin{align}
  \tilde{g}^{(n)} =& \left(\hat{A}^{n-1} + \hat{B}^{n-1} + \hat{C}^{n-1} + \hat{T}^{n-1}\right)\left(f^{(n-1)} + df^{(n-1)}\right)\\
  \tilde{g}^{(n)} =&  \bar{g}^{(n-1)} + d\bar{g}^{(n-1)}\\
  \bar{g}^{(n-1)} =&\left(\hat{C}^{n-1} + \hat{T}^{n-1}\right)\left(f^{(n-1)} + df^{(n-1)}\right)\\
  d\bar{g}^{(n-1)} =& \left(\hat{A}^{n-1} + \hat{B}^{n-1}\right)\left(f^{(n-1)} + df^{(n-1)}\right)
\end{align}
And
\begin{align}
  \begin{split}
    \hat{C}^{n-1}f^{(n-1)} \approx 0 \ ;& \ \hat{T}^{n-1}df^{(n-1)}\approx 0\\
    \bar{g}^{(n-1)} \defeq& \hat{g}^{(n-1)} + \tilde{g}^{(n-1)}\\
    \hat{g}^{(n-1)} =& \hat{C}^{n-1}df^{(n-1)}\\
    \tilde{g}^{(n-1)} =& \hat{T}^{n-1}f^{(n-1)}
  \end{split}
\end{align}
We find $\bar{g}^{(0)}$ as
\begin{equation}
  \bar{g}^{(0)} =  \tilde{g}^{(0)} + \hat{g}^{(0)} = \hat{T}^{0}f^{(0)} + \hat{C}^{0}df^{(0)}
\end{equation}
and iteratively find $\tilde{g}^{(n)}$ as \cite{Frediani:2013}
\begin{align}
  \begin{split}
    \tilde{g}^{(0)} &= \hat{T}^{0}f^{(0)} \\
    \tilde{g}^{(1)} &= \bar{g}^{(0)} + d\bar{g}^{(0)} \\
    &= \tilde{g}^{(0)} + \hat{g}^{(0)} + d\bar{g}^{(0)}\\
    &= \hat{T}^{0}f^{(0)} + \hat{C}^{0}df^{(0)} + \left(\hat{A}^{0} + \hat{B}^{0}\right)\left(f^{(0)} + df^{(0)}\right)\\
    \tilde{g}^{(2)} &= \bar{g}^{(1)} + d\bar{g}^{(1)}\\
    &= \tilde{g}^{(1)} + \hat{g}^{(1)} + d\bar{g}^{(1)} \\
    &= \bar{g}^{(0)} + d\bar{g}^{(0)} + \hat{g}^{(1)} + d\bar{g}^{(1)}\\
    &= \tilde{g}^{(0)} + \hat{g}^{(0)} + d\bar{g}^{(0)} + \hat{g}^{(1)} + d\bar{g}^{(1)}\\
    &\vdots\\
    \tilde{g}^{(N)} &= \hat{T}^{0}f^{(0)} + \sum^N_{n=0}\hat{C}^{n}df^{(n)} + \left(\hat{A}^{n} + \hat{B}^{n}\right)\left(f^{(n)}
    + df^{(n)}\right)
  \end{split}
\end{align}
And we assume that for a given refinement level $N$ we have a good approximation of
$g$.
\begin{equation}
  \tilde{g}^{(N)} \approx g^{N} = (\hat{T}f)^N
\end{equation}
\subsection{Examples of Operators}
\subsubsection{Derivative operator}

\subsubsection{Poisson Operator}

\begin{equation}\label{eq:Poissonopmw}
\hat{\mathscr{P}}(r)=\int \frac{1}{4 \pi\left\|r-r^{\prime}\right\|} f\left(r^{\prime}\right)
d r^{\prime}
\end{equation}

\section{Other Basis Sets}
\subsection{\ac{STO}}
\subsection{\ac{GTO}}

\begin{acronym}
\acro{AUS}[\href{https://www.sigma2.no/content/advanced-user-support}{AUS}]{Numerical Methods in Quantum Chemistry}
\acro{BO}{Born-Oppenheimer}
\acro{CTCC}[\href{http://www.ctcc.no}{CTCC}]{Centre for Theoretical and Computational Chemistry}
\acro{DC}{Dielectric Continuum}
\acro{DFT}{Density Functional Theory}
\acro{EFP}{Effective Fragment Potential}
\acro{EU}{European Union}
\acro{HF}{Hartree-Fock}
\acro{Hylleraas}[\href{https://www.mn.uio.no/hylleraas/english/}{Hylleraas}]{Hylleraas
  Centre for Quantum Molecular Sciences}
\acro{HPC}{High Performance Computing}
\acro{KTH}{Royal Institute of Technology}
\acro{LDA}{Local Density Approximation}
\acro{MCD}{Magnetic Circular Dichroism}
\acro{MCSCF}{Multiconfiguration Self Consistent Field}
\acro{MM}{Molecular Mechanics}
\acro{MW}{Multiwavelet}
\acro{NFR}{Norwegian Research Council}
\acro{NMQC}[\href{http://www.ctcc.no/events/conferences/2015/numeric-conference/}{NMQC}]{Numerical Methods in Quantum Chemistry}
\acro{NOTUR}[\href{https://www.notur.no/}{NOTUR}]{Norwegian Metacenter for Computational Science}
\acro{PCM}{Polarizable Continuum Model}
\acro{PI}{Primcipal Investigator}
\acro{QC}{Quantum Chemistry}
\acro{QM}{Quantum Mechanics}
\acro{QM/MM}{Quantum Mechanics/Molecular Mechanics}
\acro{ROA}{Raman Optical Activity}
\acro{SC}{semiconductor}
\acro{SCF}{Self Consistent Field}
\acro{SHG}{Second Harmonic Genertation}
\acro{STSM}{Short-term scientific mission}
\acro{TPA}{Two-Photon Absorption}
\acro{WP}{Work Package}
\acro{CBS}{Complete Basis Set}
\acro{TCG}{Theoretical Chemistry Group}
\acro{vdW}{van der Waals}
\acro{SE}{Schrödinger Equation}
\acro{PES}{Potential Energy Surface}
\acro{LCAO}{Linear Combination of Atomic Orbitals}
\acro{MRA}{Multi-Resolution Analysis}
\acro{NS}{Nonstandard}
\end{acronym}

\biblio
\end{document}

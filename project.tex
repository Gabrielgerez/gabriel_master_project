\documentclass[a4paper,11pt]{article}
\usepackage{pgfgantt}
\usepackage{graphicx}
\usepackage{hyperref}
\hypersetup{colorlinks, 
            breaklinks, 
            urlcolor=blue, 
            linkcolor=blue, 
            citecolor=blue} % Set link colors
\usepackage[left=2cm,top=2cm,bottom=2cm,right=2cm]{geometry}
\setlength{\parskip}{0.75ex}
\usepackage[utf8]{inputenc}
\usepackage[T1]{fontenc}
\usepackage[english]{babel}
\usepackage{amsmath,amssymb,amstext,amsfonts,euscript}
\usepackage[nohyperlinks,nolist]{acronym}
\usepackage[firstinits=true,backend=biber,maxbibnames=1,doi=false,isbn=false,url=false,style=numeric-comp,sorting=none]{biblatex}
%\usepackage[disable, colorinlistoftodos, textsize=small]{todonotes}
\usepackage[colorinlistoftodos, textsize=small]{todonotes}
%\renewcommand{\bibfont}{\normalfont\footnotesize}
%\defbibenvironment{bibliography}
%  {\noindent}
%  {\unspace}
%  {\printtext[bold]{\printtext[labelnumberwidth]{%
%    \printfield{prefixnumber}%
%    \printfield{labelnumber}}}
%    \addspace}
%\renewbibmacro*{finentry}{\finentry\quad}
%\renewbibmacro{in:}{}
%\DeclareFieldFormat[article]{pages}{#1}%
%\DeclareFieldFormat[article]{title}{} % uncomment to remove title in bibliography
\addbibresource{misc-refs.bib}
\addbibresource{papers.bib}
\AtEveryBibitem{\clearlist{language}}



\newcommand{\pcmsol}{\texttt{PCMSolver}}
\newcommand{\mrchem}{\href{https://mrchem.readthedocs.io/en/latest/}{\texttt{MRChem}}}
\newcommand{\mrcpp}{\href{https://mrcpp.readthedocs.io/en/latest/}{\texttt{mrcpp}}}

\newcommand{\etal}{{\emph{et al.}\ }}

\title{Combining the Polarizable Continuum Model with Multiresolution
  analysis: a cavity-free approach to solvent effects}

\author{Gabriel Adolfo Gerez}

\begin{document}

\maketitle

\section{Introduction}

One of the research activities carried out at the \ac{TCG} consists in
the development and application of \ac{DFT} methods in combination of
\ac{MW}. The developed code, called \mrchem{} is able to perform calculation of
molecular energy and properties at the \ac{CBS} limit, with
predefined and guaranteed precision, in virtue of the properties of
\ac{MW}. In order to extend the applicability of the code, a desirable
feature is the inclusion of the solvent effect. That would allow to
extend the applicability of \mrchem{} to molecules surrounded by an
environment (generally a solvent), which is most often the case in
practical applications.

\section{Methods}

\subsection{Multiwavelets}
The Schrödinger equation for a system with more than two atoms is too
complicated for an exact solution. In order to solve it we use several
approximations so that we can reach as accurate a result as possible
while mantaining reasonable computation time.  The \ac{BO}
approximation lets us divide the equation into smaller parts. We can
focus then on the electronic part of the problem.

\begin{equation}\label{eq:schrodinger}
H_e \Psi_e E_e \Psi_e
\end{equation}

 The electron wave-function must be determined to solve the
 equation. Given that it represents all the electrons in the system
 one can see that it is too complicated to solve exactly. Basis sets
 help solve this type of problems. They give a relativelly good
 approximation to the elctronic wavefunction so computation of the
 whole problem can be done more effectivelly.
 

\ac{MW} basis have several properites that are useful when designing
numerical solutions to partial differential equations. Some of these
are orhtogonality of the wavelet spaces, vanishing moments
%check more on that
and locally adaptive representations of functions, where the grid can
be refined to focus only on special places where it will refine
furhter to reach the desired precision %remember sources please.
This gives flexibility to the type of problems that can be solved by use of
this method.


I will learn how to implement this method and make adjustments to my computations to refine them.

\subsection{Density Functional Theory}

Hohenberg and Kohn %source
demonstrated that the
electronic energy ($E_e$ in equation 1) is completelly determined by
the electron density $\rho$\cite{parr1994density}. As outlined by E. B. Wilson, %sources
the integral of the
density determines the number of electrons (for a one elctron system
the integral is equal to 1), the cusps and their heights define the
position of the nuclei and the amount of nucleus charges
respectivelly.  Thus to solve for the energy of the system one only
need to know the electron density and the functional that describes
the energy in terms of the density. Since the electron density always
depends on the same pamount of variables regardless of the amount of
electrons it can make computations less heavy. the only problem is
finding the right functional.


\subsection{Solvent effect}

The goal with theorethical chemistry is to simulate chemical processes
with as good accuracy as possible (relative to experimental
values). It is possible to compute a system of a molecule in vacuum to
a relative accurate level. This is not sufficient as most processes in
chemistry happen in solvents, especially reactions. If one is to
simulater such systems, the molecule in a vacuum model is not enough,
the solvent effect has to be taken into account.

One way would be to try and compute the energy of the molecules in
question together with a big enough sample of the solvent moelcules
surrounding it. This becomes quite heavy and is not feasible anymore,
moreover, it is not certain that this would simulate the solvent
effect accuratelly.

One of the used models it to make a solvent continuum. This mean to
make an average of the interactions the solvent has as a whole and
model it as a continuum. The next step is to put the molecule to be
analyzed in this continuum. There are several ways to do this (MM/QM
methods and different surface descriptions). The method i will be
using at the start of this project is The interlocking spheres model.

The interlocking spheres model can be explained as creating spheres
around the atoms of the molecule. The radii of the spheres are equal
to the van der Waals radius of each atom. these spheres interlock with
each other, hence the name of the model, and make a cavity where the
continuum of the solvent can't be in.  The cavity model I'm using in
this project describes the cavity as having a value of 1 inside the
cavity and 0 outside the cavity. The function describing the cavity is
as follows.  The signed distance of an arbitrary point $r$ from the
surface of the Interlocking-sphere surface with atom positions $r_i$
and \ac{vdW} radii $R_i$ is described by the following function
\begin{equation}
s_i(r) = \vert r - r_i\vert - R_i
\end{equation}
The sphere surrounding the ith atom is described by the following characteristic function.
\begin{equation}
C_i(r) = 1 - \Theta (s_i(r)) 
\end{equation}
Where $\Theta$ is the heavyside function, which returns 1 inside the
cavity ($C$ in the function 3) 0 outside and $1/2$ in the
surface. $\Theta$ is defined as a heavyside-like function that
smoothly switches over a controllable shell width $\sigma$. This makes
it easier to control the transition, so that it isn't as abrupt.
\begin{equation}
\Theta (s/\sigma) = \frac{1}{2} (1 + erf(s/\sigma))
\end{equation}
For an amount of atoms $N$ we can then describe the interlocking spheres with the following function.
\begin{equation}
C(r) = 1 - \prod_{i=1}^{N} 1 - C_i (r)
\end{equation}

This would describe the spheres around the atoms with a value of 1
inside and a value of 0 inside. What is needed to compute the solvent
effect is the cavity describing the molecule in the solvent continuum
with a value of $\epsilon_0$ (permitivity of free space with value
1.0) inside and a value of $\epsilon_\infty$ (dielectric constant of
the solvent, ca. 80 for water). Using these constants and the cavity
function $C$, the following dielectric function is computed.
\begin{equation}
\epsilon (r) = \epsilon_0 C(r) + \epsilon (1 - C(r))
\end{equation}


\subsection{Project goal}

I will provide \mrchem{} with the
framework to include environmental effect. Currently such a capability
is present in a separate branch of the code and shall be transferred
to the master branch to make it available for applications.

\subsection{Plan and milestones}

The master project will be divided into the following activities.

\subsubsection{Literature search and theoretical development}

To carry out the project successfully, I will acquire the relevant
background in the following topics: \acl{DFT}, \aclp{MW}, and \acl{DC}
methods.
\todo{here you should list the references...}

\subsubsection{Python implementation}

I will implement the methods necessary for this project using Python.
I am currenctly working on a python code to solve a solvation example
for a Hydrogen atom. I will also assist in development of the 'vampyr'
branch of \mrcpp{} which will be merged with the main branch of the group
later.

\subsubsection{C++ implementation}
I am adding methods to the vampyr.cpp part of the \mrcpp{} code so
that it can be used with python. I will also read through the work
done on the subect i am working on. The \mrcpp{} package is based
around the numerical \aclp{MW} representations of functions and
operators.

\subsubsection{Variational formulation}

A recent development of \ac{PCM}, has made it possible to include the
solvent polarization in the standard
\ac{SCF} procedure, by recasting the electrostatic polarization
problem as a minimization procedure for a solvation energy
functional\cite{Lipparini:2010bg}. Such a formulationis is 
ideal for a
\ac{MW} implementation, where the solvent response is described by one
single function, on top of the occupied molecular
orbitals. The implementation, which
will require one loop only instead of two nested loops, will be simplified.

\subsubsection{Density-dependent cavity}

The cavity that is described by the interlocking spheres created by
the position of the atoms and their \ac{vdW} radius. This will be
developed such that this cavity can be represented by the electron
density instead, thus making the cavity more dynamic to changes and
forces acting on the molecule. This will give us a density dependent
dielectric function.

\subsubsection{Molecular properties}

Once the framework for the solvation effect is in place, we will apply
it to the calculation of molecular properties. Contrary to standard
solvation methods based on the solution of a boundary equation, the
direct solution of the generalizaed Poisson equation usd here will allow to
account for escaped charge effects, which can be important for diffuse
excited states such as Rydberg excitations.

\begin{figure}[!htb]
\begin{ganttchart}
[
x unit = 10mm,
y unit chart = 5mm,
y unit title = 6mm,
hgrid, vgrid,
milestone/.append style={fill=orange, rounded corners=3pt},
canvas/.style=%
{shape=rectangle, fill=yellow!25,
draw=blue, dashed, very thick}
]
{1}{9}
\gantttitle{2018}{4} 
\gantttitle{2019}{5} \\
\gantttitlelist{9,10,11,12,1,2,3,4,5}{1} \\
\ganttgroup{Literature study}{1}{6} \\
\ganttgroup{Python Implementation}{1}{2} \\
\ganttgroup{C++ implementation }{2}{4} \\
\ganttgroup{Variational Formulation}{4}{7} \\
\ganttgroup{Molecular properties}{5}{9} \\
\end{ganttchart}
  \caption{The timeline of the project illustrating the expected
    start and completion of each milestone}
  \label{fig:gantt}
\end{figure}


\subsection{Tools and Methods}

This is a computational project which will mainly consist in the
development of a code to run calculations combining \ac{DFT},
\acp{MW} and continuum solvation. The implementation will be carried
out both in \mrcpp{} and in \mrchem, and the calculations will be
performed on the \ac{HPC} infrastructure of UiT, in particular on the ``Stallo'' supercomputer.

\subsection*{References}
%\footnotesize
\printbibliography[heading=none]


\begin{acronym}
\acro{AUS}[\href{https://www.sigma2.no/content/advanced-user-support}{AUS}]{Numerical Methods in Quantum Chemistry}
\acro{BO}{Born-Oppenheimer}
\acro{CTCC}[\href{http://www.ctcc.no}{CTCC}]{Centre for Theoretical and Computational Chemistry}
\acro{DC}{Dielectric Continuum}
\acro{DFT}{Density Functional Theory}
\acro{EFP}{Effective Fragment Potential}
\acro{EU}{European Union}
\acro{HF}{Hartree-Fock}
\acro{Hylleraas}[\href{https://www.mn.uio.no/hylleraas/english/}{Hylleraas}]{Hylleraas
  Centre for Quantum Molecular Sciences}
\acro{HPC}{High Performance Computing}
\acro{KTH}{Royal Institute of Technology}
\acro{LDA}{Local Density Approximation}
\acro{MCD}{Magnetic Circular Dichroism}
\acro{MCSCF}{Multiconfiguration Self Consistent Field}
\acro{MM}{Molecular Mechanics}
\acro{MW}{Multiwavelet}
\acro{NFR}{Norwegian Research Council}
\acro{NMQC}[\href{http://www.ctcc.no/events/conferences/2015/numeric-conference/}{NMQC}]{Numerical Methods in Quantum Chemistry}
\acro{NOTUR}[\href{https://www.notur.no/}{NOTUR}]{Norwegian Metacenter for Computational Science}
\acro{PCM}{Polarizable Continuum Model}
\acro{PI}{Primcipal Investigator}
\acro{QC}{Quantum Chemistry}
\acro{QM}{Quantum Mechanics}
\acro{QM/MM}{Quantum Mechanics/Molecular Mechanics}
\acro{ROA}{Raman Optical Activity}
\acro{SC}{semiconductor}
\acro{SCF}{Self Consistent Field}
\acro{SHG}{Second Harmonic Genertation}
\acro{STSM}{Short-term scientific mission}
\acro{TPA}{Two-Photon Absorption}
\acro{WP}{Work Package}
\acro{CBS}{Complete Basis Set}
\acro{TCG}{Theoretical Chemistry Group}
\acro{vdW}{van der Waals}
\acro{SE}{Schrödinger Equation}
\acro{PES}{Potential Energy Surface}
\acro{LCAO}{Linear Combination of Atomic Orbitals}
\acro{MRA}{Multi-Resolution Analysis}
\acro{NS}{Nonstandard}
\end{acronym}


\end{document}


